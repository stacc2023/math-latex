\usepackage[landscape, margin=0.5cm]{geometry}
\usepackage{multicol}

\usepackage{helvet} % Use Helvetica font (similar to Arial)
\renewcommand{\familydefault}{\sfdefault} % 기본 글꼴을 sans-serif로 설정

\usepackage{amsmath, amsthm}

% Define a new theorem style
\newtheoremstyle{definition}
{1em} % Space above
{0.5em} % Space below
{\normalfont} % Body font
{} % Indent amount
{\bfseries} % Theorem head font
{} % Punctuation after theorem head
{0cm} % Space after theorem head
{\llap{\thmname{#1} \thmnumber{#2}\hskip1em}\thmnote{#3\vspace{0.5em}\newline}}

% Apply the new style to the definition environment
% \theoremstyle{definition}
\theoremstyle{definition}


% Define a custom definition environment
\newtheorem{definition}{Definition}[section]
\newtheorem{theorem}[definition]{Theorem}
\newtheorem{example}[definition]{Example}
\newtheorem{exercise}{Exercise}[section]
\newtheorem*{remark}{Remark}
\newtheorem*{corollary}{Corollary}

% Customize the section style
\usepackage{titlesec}
% \titleformat{\section}[block]
% {\normalfont\large\bfseries}
% {\llap{Section \thesection.0\hskip1em}}{0pt}{}

\renewcommand{\thesubsection}{\arabic{subsection}}
\titleformat{\subsection}[block]
{\normalfont\normalsize\bfseries}
{\llap{Concept \thesubsection\hskip1em}}{0pt}{\large}



% enumerate
\usepackage{enumitem}
% \alph*: 소문자 알파벳 (a, b, c, ...)
% \Alph*: 대문자 알파벳 (A, B, C, ...)
% \roman*: 소문자 로마 숫자 (i, ii, iii, ...)
% \Roman*: 대문자 로마 숫자 (I, II, III, ...)

% math font
\usepackage{amsfonts}


% Redefine the proof environment to match the theorem style
\makeatletter
\renewenvironment{proof}[1][(pf)]{%
    \par
    \pushQED{\qed}%
    \normalfont\topsep3pt\relax
    % \normalfont \topsep6\p@\@plus6\p@\relax
    \trivlist
    \item[\llap{\bfseries#1\hskip0em}]
    \leftskip=2.5em
    \def\forward{\item[\llap{\bfseries($\Rightarrow$)\hskip0em}]}
    \def\backward{\item[\llap{\bfseries($\Leftarrow$)\hskip0em}]}
    \newcommand{\step}[1]{\@ifnextchar\bgroup{\step@binary{##1}}{\step@single{##1}}}
    \newcommand{\step@binary}[2]{\item[\llap{(##1) $\Rightarrow$ (##2)\hskip0em}]}
    \newcommand{\step@single}[1]{\item[\llap{(##1)\hskip0em}]}
}{%
    \popQED\endtrivlist\@endpefalse
}
\makeatother

\usepackage{xcolor} % color
\definecolor{softred}{RGB}{239, 41, 41} % softred


\makeatletter
\newenvironment{hardproof}[1][\textcolor{softred}{(pf)}]{%
    \par
    \pushQED{\qed}%
    \normalfont\topsep3pt\relax
    % \normalfont \topsep6\p@\@plus6\p@\relax
    \trivlist
    \item[\llap{\bfseries#1\hskip0em}]
    \leftskip=2.5em
    \def\forward{\item[\llap{\bfseries($\Rightarrow$)\hskip0em}]}
    \def\backward{\item[\llap{\bfseries($\Leftarrow$)\hskip0em}]}
    \newcommand{\step}[1]{\@ifnextchar\bgroup{\step@binary{##1}}{\step@single{##1}}}
    \newcommand{\step@binary}[2]{\item[\llap{(##1) $\Rightarrow$ (##2)\hskip0em}]}
    \newcommand{\step@single}[1]{\item[\llap{(##1)\hskip0em}]}
}{%
    \popQED\endtrivlist\@endpefalse
}
\makeatother


% hyperlink
\usepackage{hyperref}

\newcommand{\diam}{\text{diam }}


% because
\usepackage{amssymb}


% box
\usepackage[most]{tcolorbox}

\newtcolorbox{notebox}[1][]{
    colback=white!0,
    colframe=black,
    sharp corners,
    boxrule=1pt,
    valign=top,
    left=5pt,
    #1
}

%integral
\usepackage{amsmath}

\def\upint{\mathchoice%
    {\mkern13mu\overline{\vphantom{\intop}\mkern7mu}\mkern-20mu}%
    {\mkern7mu\overline{\vphantom{\intop}\mkern7mu}\mkern-14mu}%
    {\mkern7mu\overline{\vphantom{\intop}\mkern7mu}\mkern-14mu}%
    {\mkern7mu\overline{\vphantom{\intop}\mkern7mu}\mkern-14mu}%
  \int}
\def\lowint{\mkern3mu\underline{\vphantom{\intop}\mkern7mu}\mkern-10mu\int}

%Riemann integral
\usepackage{mathrsfs}

% footnote horizontal
\usepackage[para]{footmisc}

\usepackage[most]{tcolorbox}
\usepackage{xcolor}

\definecolor{mintgreen}{HTML}{E0F8E0}
\definecolor{skyblue}{HTML}{D6EBFF}
\definecolor{peach}{HTML}{FFE5B4}


\newtcolorbox[use counter=definition, number within=section]{mydefinition}[1][]{
  colback=peach,        % 배경색
  colframe=peach,       % 경계선 색상 (배경색과 동일하게 설정하여 경계선을 없앰)
  fonttitle=\bfseries,         % 제목의 글씨체를 볼드체로
  title={\thetcbcounter \, Definition%
         \ifstrempty{#1}{}{:\ #1}}, % 제목 서식 설정, 인수 #1이 비어 있지 않으면 ": #1" 추가
  boxrule=0pt,                 % 경계선 두께 설정 (0pt로 설정하여 경계선 없음)
  sharp corners,               % 모서리를 뾰족하게
  coltitle=black,              % 제목 색상
  colbacktitle=peach,      % 제목 배경색 (본문 배경색과 동일)
  enhanced,                    % 박스의 기타 그래픽적 요소들 향상
  toptitle=5pt,
  left=5pt,
  right=5pt,
  bottom=5pt,
}


\newtcolorbox[use counter=definition, number within=section]{mytheorem}[1][]{
  colback=skyblue,
  colframe=skyblue,
  fonttitle=\bfseries,
  title={\thetcbcounter \, Theorem%
  \ifstrempty{#1}{}{:\ #1}},
  boxrule=0pt,
  sharp corners,
  coltitle=black,
  colbacktitle=skyblue,
  enhanced,                    % 박스의 기타 그래픽적 요소들 향상
  toptitle=5pt,
  left=5pt,
  right=5pt,
  bottom=5pt,
}

\newtcolorbox[use counter=definition, number within=section]{mylemma}[1][]{
    colback=mintgreen,
    colframe=mintgreen,
    fonttitle=\bfseries,
    title={\thetcbcounter \, Lemma%
    \ifstrempty{#1}{}{:\ #1}},
    boxrule=0pt,
    sharp corners,
    coltitle=black,
    colbacktitle=mintgreen,
    enhanced,                    % 박스의 기타 그래픽적 요소들 향상
    toptitle=5pt,
    left=5pt,
    right=5pt,
    bottom=5pt,
}

\newtcolorbox[use counter=definition, number within=section]{myproposition}[1][]{
    colback=mintgreen,
    colframe=mintgreen,
    fonttitle=\bfseries,
    title={\thetcbcounter \, Proposition%
    \ifstrempty{#1}{}{:\ #1}},
    boxrule=0pt,
    sharp corners,
    coltitle=black,
    colbacktitle=mintgreen,
    enhanced,                    % 박스의 기타 그래픽적 요소들 향상
    toptitle=5pt,
    left=5pt,
    right=5pt,
    bottom=5pt,
}

\newtcolorbox[use counter=definition, number within=section]{mycorollary}[1][]{
    colback=mintgreen,
    colframe=mintgreen,
    fonttitle=\bfseries,
    title={\thetcbcounter \, Corollary%
    \ifstrempty{#1}{}{:\ #1}},
    boxrule=0pt,
    sharp corners,
    coltitle=black,
    colbacktitle=mintgreen,
    enhanced,                    % 박스의 기타 그래픽적 요소들 향상
    toptitle=5pt,
    left=5pt,
    right=5pt,
    bottom=5pt,
}

\newtheoremstyle{definition}
{0.5em} % Space above
{0.5em} % Space below
{\normalfont} % Body font
{} % Indent amount
{\bfseries} % Theorem head font
{} % Punctuation after theorem head
{1em} % Space after theorem head
{}

\titleformat{\section}[block]
{\normalfont\large\bfseries}
{Section \thesection. }{0pt}{}

\renewcommand{\thesubsection}{\arabic{subsection}}
\titleformat{\subsection}[block]
{\normalfont\normalsize\bfseries}
{}{0pt}{\large}

\makeatletter
\renewenvironment{proof}[1][\textbf{\proofname)}]{%
    \par
    \pushQED{\qed}%
    \normalfont\topsep3pt\relax
    % \normalfont \topsep6\p@\@plus6\p@\relax
    \trivlist
    \item #1
    \def\forward{\item[\llap{\bfseries($\Rightarrow$)\hskip0em}]}
    \def\backward{\item[\llap{\bfseries($\Leftarrow$)\hskip0em}]}
    \newcommand{\step}[1]{\@ifnextchar\bgroup{\step@binary{##1}}{\step@single{##1}}}
    \newcommand{\step@binary}[2]{\item[\llap{(##1) $\Rightarrow$ (##2)\hskip0em}]}
    \newcommand{\step@single}[1]{\item[\llap{(##1)\hskip0em}]}
}{%
    \popQED\endtrivlist\@endpefalse
}
\makeatother

% img
\usepackage{graphicx}
\usepackage{wrapfig}
\usepackage{capt-of}