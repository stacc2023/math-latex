% ---------- ---------- ---------- ---------- ----------
% ---------- ---------- Section 12 ---------- ----------
% ---------- ---------- ---------- ---------- ----------
\subsection{12. Topological space}

% def of topology
\begin{mydefinition}
[Topology]
A \textbf{\emph{topology}} on a set $X$ is a set of $\mathcal{T}$ of subsets of $X$ having the following properties:
\begin{enumerate}[label={(\alph*)}]
\item $\varnothing ,X\in \mathcal{T}$.
\item The union of any subset of $\mathcal{T}$ is in $\mathcal{T}$.
\item The intersection of any  finite subset of $\mathcal{T}$ is in $\mathcal{T}$.
\end{enumerate}
A set $X$ for which a topology $\mathcal{T}$ has been specified is called a \textbf{\emph{topological space}}. We say $U\subset X$ is an \textbf{\emph{open set}} if $U\subset \mathcal{T}$.
\end{mydefinition}
% tivial topology, discrete topology
\begin{example}
A topology $\mathcal{T}=\{\varnothing, X\}$ is called the \textbf{\emph{indiscrete topology}} or \textbf{\emph{trivial topology}}. On the other hand, if $\mathcal{T}=P(X)$, then it is called the \textbf{\emph{discrete topology}}
\end{example}

\begin{example}
Let $X$ ba a set, $\mathcal{T}_f$ be a set of all subset $U\subset X$ such that $X\backslash U$ either is finite or is all of $X$. Then
\begin{enumerate}[label={(\alph*)}]
\item $X\backslash X$ is finite and $X\backslash \varnothing$ is $X$.
\item Suppose $\{U_\alpha\}$ is an indexed family of nonempty elements of $\mathcal{T}_f$. Then $X-\bigcup U_\alpha = \bigcap(X-U_\alpha)$ is finite.
\item $X-\bigcap_{i=1}^n = \bigcup_{i=1}^n(X-U_\alpha)$ is finite.
\end{enumerate}
Therefore, $\mathcal{T}_f$ is a topology, called \textbf{\emph{finite complement topology}}. We can replace the condition 'finite' with 'countable', and the proposition still holds.
\end{example}

\begin{mydefinition}
Suppose $\mathcal{T}$ and $\mathcal{T}'$ are two topologies on a given set $X$. If $\mathcal{T}'\supset \mathcal{T}$, then we say that $\mathcal{T}'$ is \textbf{\emph{finer(larger)}} than $\mathcal{T}$, and $\mathcal{T}$ is \textbf{\emph{coarser(smaller)}} than $\mathcal{T}'$. Also, we say that $\mathcal{T}$ is \textbf{\emph{comparable}} with $\mathcal{T}'$ if either $\mathcal{T}\supset \mathcal{T}'$ or $\mathcal{T}'\supset \mathcal{T}$.
\end{mydefinition}

% ---------- ---------- ---------- ---------- ----------
% ---------- ---------- Section 13 ---------- ----------
% ---------- ---------- ---------- ---------- ----------
\subsection{13. Basis for a topology}
\begin{mydefinition}
If $X$ is a set, a \textbf{\emph{basis}} for a topology on $X$ is a collection $\mathcal{B}$ of subsets of $X$(called a \textbf{\emph{basis elements}}) such that
\begin{enumerate}[label={(\alph*)}]
\item For each $x\in X$, there is at least one basis element $B$ containing $x$.
\item If $x$ belongs to the intersection of two basis elements $B_1$ and $B_2$, then there is a basis element $B_3$ containing $x$ such that $B_3\subset B_1\cap B_2$. We deine the \textbf{\emph{topology $\mathcal{T}$ generated by $\mathcal{B}$}} as follows: A subset $U$ of $X$ is said to be open in $X$ if for each $x\in U$, there is a basis element $B\in \mathcal{B}$ such that $x\in B$ and $B\subset U$.
\end{enumerate}
\end{mydefinition}

\begin{example}
Consider the set of all rectangular regions in $\mathbb{R}^2$ without border, where each rectangular have sides parallel to the coordinate axes. 
\end{example}

\begin{mylemma}
A topology is equal to the set of all union of its basis elements.
\end{mylemma}
\begin{proof}
[Munkres lemma 13.1]
\end{proof}

\begin{mylemma}
Let $X$ be a topological space. Suppose that $\mathcal{C}$ is a set of open sets of $X$ such that for each $U\underset{\text{open}}{\subset} X$ and each $x\in U$, there is an element $C\in \mathcal{C}$ such that $x\in C\subset U$. Then $\mathcal{C}$ is a basis for the topology of $X$.
\end{mylemma}
\begin{proof}
[Munkres lemma 13.2]
\end{proof}

\begin{mylemma}
Let $\mathcal{B}$ and $\mathcal{B}'$ be bases for the topologies $\mathcal{T}$ and $\mathcal{T}'$, respectively, on $X$. TFAE:
\begin{enumerate}[label={(\alph*)}]
\item $\mathcal{T}'$ is finer than $\mathcal{T}$.
\item For each $x\in X$ and each $B\in \mathcal{B}$ containing $x$, there is a basis element $B'\in \mathcal{B}'$ such that $x\in B'\subset B$.
\end{enumerate}
\end{mylemma}
\begin{proof}
[Munkres lemma 13.3]
\end{proof}

\begin{mydefinition}
In $\mathbb{R}$,
\begin{enumerate}[label={(\alph*)}]
\item The set of all open interval is a basis of the \textbf{\emph{standard topology}} on $\mathbb{R}$.
\item The set of all half-open($[a,b)$) intervals is a basis of the \textbf{\emph{lower limit topology}} on $\mathbb{R}$, denoted by $\mathbb{R}_l$.
\item Let $K=\{1/n\mid n\in \mathbb{Z}\}$. The set of all open interval without the points in $K$ is a basis of the \textbf{\emph{K-topology}} on $\mathbb{R}$, denoted by $\mathbb{R}_K$.
\end{enumerate}
\end{mydefinition}

\begin{mylemma}
$\mathbb{R}_l$ and $\mathbb{R}_K$ are strictly finer than the standard topology on $\mathbb{R}$, but are not comparable with one another.
\end{mylemma}
\begin{proof}
[Munkres lemma 13.4]
\end{proof}

\begin{mydefinition}
A \textbf{\emph{subbasis}} $\mathcal{S}$ for a topology on $X$ is a collection of subsets of $X$ whose union equals $X$. The \textbf{\emph{topology generated by the subbasis}} $\mathcal{S}$ is defined to be the set $\mathcal{T}$ of all unions of finite intersections of elements of $\mathcal{S}$.
\end{mydefinition}

% ---------- ---------- ---------- ---------- ----------
% ---------- ---------- Section 14 ---------- ----------
% ---------- ---------- ---------- ---------- ----------
\subsection{14. The order topology}
\begin{mydefinition}
If $X$ is a set with the relation $<$, the topology derived from the basis $\mathcal{B}$ of subset of $X$ such that
\begin{enumerate}[label={(\alph*)}]
\item All open intervals $(a,b)$ in $X$;
\item all intervals of the form $[a_0,b)$ in $X$(if the smallest element $a_0$ exists);
\item all intervals of the form $[a,b_0]$ in $X$(if the largest element $b_0$ exists),
\end{enumerate}
is called the \textbf{\emph{order topology}}.
\end{mydefinition}
\begin{example}
The standard topology on $\mathbb{R}$ is a order topology.
\end{example}
\begin{example}
The order topology on $\mathbb{Z}^+$ is the discrete topology.
\end{example}
\begin{example}
The basis elements for order topology of the set $\mathbb{R}\times \mathbb{R}$ in the dictionary order is of the form $(a\times b, c\times d)$ for $a<c$ or $a=c$ and $b<d$.
\end{example}
\begin{example}
The order topology of the set $X=\{1,2\}\times \mathbb{Z}^+$ in the dictionary order is not the discrete topology. Consider the basis element containing $(2,1)$.
\end{example}

\begin{mydefinition}
Suppose $X$ is an ordered set, and $a\in X$. There are four subsets of $X$:
\begin{enumerate}[label={(\alph*)}]
\item $(a,+\infty)=\{x\mid x>a\}$.
\item $(-\infty,a)=\{x\mid x<a\}$.
\item $[a,+\infty]=\{x\mid x\geq a\}$.
\item $(-\infty,a)=\{x\mid x\leq a\}$.
\end{enumerate}
They are called the \textbf{\emph{rays}} determined by $a$. Sets of the first two types are called \textbf{\emph{open rays}}, and sets of the last two types are called \textbf{\emph{closed rays}}. The open rays form a subbasis for the order topology on $X$.
\end{mydefinition}

% ---------- ---------- ---------- ---------- ----------
% ---------- ---------- Section 15 ---------- ----------
% ---------- ---------- ---------- ---------- ----------
\subsection{15. The product topology on $X\times Y$}
\begin{mydefinition}
Let $X,Y$ be topological spaces. The \textbf{\emph{product topology}} on $X\times Y$ is the topology having as basis the set $\mathcal{B}$ of all sets of the form $U\times V$, where $U\underset{\text{open}}{\subset}X$ and $V\underset{\text{open}}{\subset}Y$. 
\end{mydefinition}

\begin{remark}
$\mathcal{B}$ is not a topology on $X\times Y$.
\end{remark}

\begin{mytheorem}
If $\mathcal{B}$ is a basis for the topology of $X$ and $\mathcal{C}$ is a basis for the topology of $Y$, then the set $\mathcal{D}=\{B\times C\mid B\in \mathcal{B},C\in \mathcal{C}\}$ is a basis for the topology of $X\times Y$.
\end{mytheorem}
\begin{proof}
[Munkres theorem 15.1]
\end{proof}

\begin{mydefinition}
Define $\pi_1:X\times Y\to X$ by the equation $\pi_1(x,y)=x$, and define $\pi_2:X\times Y\to Y$ by $\pi_2(x,y)=y$. The maps are called the \textbf{\emph{projection}} of $X\times Y$ onto its first and second factors, respectively.
\end{mydefinition}

\begin{mytheorem}
$\mathcal{S}=\{\pi_1^{-1}(U)\mid U\underset{\text{open}}{\subset}X\} \cup \{\pi_2^{-1}(V)\mid V\underset{\text{open}}{\subset}Y\}$ is a subbasis for the product topology on $X\times Y$.
\end{mytheorem}
\begin{proof}
[Munkres theorem 15.2]
\end{proof}

% ---------- ---------- ---------- ---------- ----------
% ---------- ---------- Section 16 ---------- ----------
% ---------- ---------- ---------- ---------- ----------
\begin{mydefinition}
Let $X$ be a topological space with a topology $\mathcal{T}$. If $Y\subset X$, $\mathcal{T}_Y=\{Y\cap U\mid U\in \mathcal{T}\}$ is a topology on $Y$, called the \textbf{\emph{subspace topology}}. With this topology, $Y$ is called a subspace of $X$.
\end{mydefinition}

\begin{mylemma}
If $\mathcal{B}$ is a basis for the topology on $X$, then the set $\mathcal{B}_Y=\{B\cap Y\mid B\in \mathcal{B}\}$ is a basis for the subspace topology on $Y$.
\end{mylemma}
\begin{proof}
[Munkres lemma 16.1]
\end{proof}

\begin{notebox}
We say $U\subset Y$ is \textbf{\emph{open in(or open relative to)}}$Y$ if $U\subset \mathcal{T}_Y$.
\end{notebox}

\begin{mylemma}
Let $Y$ be a subspace of $X$. If $U$ is open in $Y$ and $Y$ is open in $X$, then $U$ is open in $X$.
\end{mylemma}
\begin{proof}
Munkres lemma 16.2
\end{proof}

\begin{mytheorem}
Suppose $A$ is a subspace of $X$ and $B$ is a subspace of $Y$. Then the product topology of $A\times B$ is same as the subspace topology of $A\times B$.
\end{mytheorem}
\begin{proof}
Munkres lemma 16.3
\end{proof}

\begin{example}
Suppose $Y=[0,1]$ is a subspace of $\mathbb{R}$. Then the subspace topology of $Y$ and the order topology on $[0,1]$ are same.
\end{example}
\begin{example}
Let $Y=[0,1)\cup \{2\}$. If $Y$ is a subspace of $\mathbb{R}$, then $\{2\}$ is open in the subspace topology on $Y$(since $(3/2,5/2)\cap \{2\}=\{2\}$). But $\{2\}$ is not open in the order topology on $Y$(since $\{2\}$ must be in the set of the form $(a,2]$).
\end{example}
\begin{example}
Let $I=[0,1]$ and $I\times I$ be of the dictionary order on $\mathbb{R}\times \mathbb{R}$. Then its subspace topology and its order topology are not same. Consider the set $\{1/2\}\times (1/2,1]$, which is open in $I\times I$ in the subspace topology, but not in order topology.
\end{example}

\begin{mytheorem}
Let $X$ be an ordered set in the order topology, and let $Y$ be a subset of $X$ that is convex in $X$. Then the order topology on $Y$ is the same as the topology $Y$ inherits as a subspace of $X$.
\end{mytheorem}
\begin{proof}
[Munkres theorem 16.4]
\end{proof}

% ---------- ---------- ---------- ---------- ----------
% ---------- ---------- Section 17 ---------- ----------
% ---------- ---------- ---------- ---------- ----------
\subsection{17. Closed sets and limit points}
\begin{mydefinition}
Let $X$ be a topological space with topology $\mathcal{T}$. By a \textbf{\emph{closed}} set, we mean a subset $A$ of $X$ that is $A=X-B$ for some $B\in \mathcal{T}$.
\end{mydefinition}
\begin{example}
In the discrete topology on the set $X$, every set is open and closed.
\end{example}
\begin{example}
Consider the set $Y=[0,1]\cup (2,3)$. Both interval $[0,1]$ and $(2,3)$ are open and closed.
\end{example}

\begin{mytheorem}
Let $X$ be a topological space. Then
\begin{enumerate}[label={(\alph*)}]
\item $\varnothing$ and $X$ are closed.
\item Arbitrary intersections of closed sets are closed.
\item Finite unions of closed sets are closed.
\end{enumerate}
\end{mytheorem}
\begin{proof}
Clear by the definition of open set and DeMorgan's law.
\end{proof}

\begin{mytheorem}
Let $Y$ be a subspace of $X$. Then a set $A$ is \textbf{\emph{closed in}} $Y$ if and only if it equals the intersection of a closed set of $X$ with $Y$.
\end{mytheorem}
\begin{proof}
Munkres theorem 17.2
\end{proof}

\begin{myproposition}
Let $Y$ be a subspace of $X$. If $A$ is closed in $Y$ and $Y$ is closed in $X$, then $A$ is closed in $X$.
\end{myproposition}

\begin{mydefinition}
Given a subset $A$ of a topological space $X$, the \textbf{\emph{interior}}(denoted by $\text{Int} A$) of $A$ is defined as the union of all open sets contained in $A$, and the \textbf{\emph{closure}}(denoted by $\text{Cl} A$ or $\overline{A}$) of $A$ is defined as the intersection of all closed sets containing $A$.
\end{mydefinition}

\begin{mytheorem}
Let $Y$ be a subspace of $X$, let $A$ be a subset of $Y$, let $\overline{A}$ denote the closure of $A$ in $X$. Then the closure of $A$ in $Y$ equals $\overline{A}\cap Y$.
\end{mytheorem}
\begin{proof}
Munkres theorem 17.5
\end{proof}

\begin{notebox}
$A$ \textbf{\emph{intersects}} $B$ if $A\cap B\neq \varnothing$.
\end{notebox}

\begin{mytheorem}
Let $A$ be a subset of the topological space $X$. 
\begin{enumerate}[label={(\alph*)}]
\item $x\in \overline{A} \iff$ every open set $U$ containing $x$ intersects $A$.
\item $x\in \overline{A} \iff$ every basis element $B$ containing $x$ intersects $A$.
\end{enumerate}
\end{mytheorem}
\begin{proof}
Munkres p96
\end{proof}

\begin{notebox}
We can shorten the statement "$U$ is an open set containing $x$" to the phrase "$U$ is a \textbf{\emph{neighborhood}} of $x$".
\end{notebox}

\begin{mydefinition}
Let $X$ be a topological space, and suppose $x\in A\subset X$. Then $x$ is called a \textbf{\emph{limit(or cluster or accumulation) point}} of $A$ if every neighborhood of $x$ intersects $A$ in some points other than $x$. $A'$ denote the set of all limit points of $A$.
\end{mydefinition}

\begin{mytheorem}
Let $X$ be a topological space, and suppose $A\subset X$. Then $\overline{A}=A\cup A'$.
\end{mytheorem}
\begin{proof}
Munkres theorem 17.6
\end{proof}

\begin{mycorollary}
Let $X$ be a topological space, and suppose $A\subset X$. Then $A$ is closed $\iff A'\subset A$.
\end{mycorollary}
\begin{proof}
Munkres corollary 17.7 
\end{proof}

\begin{mydefinition}
If $X$ is a topological space, a sequence of points of $X$ is called 
\textbf{\emph{converge}} to $x\in X$ if for each neighborhood $U$ of $x$, there exists a positivie integer $N$ such that $x_n\in U$ whenever $n\geq N$. 
\end{mydefinition}

\begin{remark}
In general topological space,
\begin{enumerate}[label={(\alph*)}]
\item A one point set in the space need not be closed.
\item A sequence of points of the space can converge to more than one point.
\end{enumerate}
\end{remark}

\begin{mydefinition}
A topological space $X$ is called a \textbf{\emph{Hausdorff space}} if for each pair $x_1$, $x_2$ of distinct points of $X$, there exist neighborhoods $U_1$, and $U_2$ of $x_2$, respectively, that are disjoint.
\end{mydefinition}

\begin{mytheorem}
Every finite point set in a Hausdorff space $X$ is closed.
\end{mytheorem}
\begin{proof}
Munkres theorem 17.8
\end{proof}

\begin{mytheorem}
Let $X$ be a space satisfying the $T_1$ axiom(that is, finite set in $X$ closed); let $A$ be a subset of $X$. Then the point $x$ is a limit point of $A$ if and only if every neighborhood of $x$ contains infinitely many points of $A$.
\end{mytheorem}
\begin{proof}
Munkres theorem 17.9
\end{proof}

\begin{mytheorem}
If $X$ is a Hausdorff space, then a sequence of points of $X$ converges to at most one point of $X$.
\end{mytheorem}
\begin{proof}
Munkres theorem 17.10
\end{proof}

\begin{mydefinition}
If a sequence $x_n$ of points of the Hausdorff space $X$ converges to the point $x$ of $X$, we write $x_n\to x$, and we say that $x$ is the \textbf{\emph{limit}} of th sequence $x_n$.
\end{mydefinition}

\begin{mytheorem}
\begin{enumerate}[label={(\alph*)}]
\item Every simply ordered set is a Hausdorff set space in the order topology.
\item The product of two Hausdorff spaces is a Hausdorff space.
\item A subspace of a Hausdorff space is a Hausdorff space
\end{enumerate}  
\end{mytheorem}
\begin{proof}
Exercise.
\end{proof}

% ---------- ---------- ---------- ---------- ----------
% ---------- ---------- Section 22 ---------- ----------
% ---------- ---------- ---------- ---------- ----------
\subsection{22. The quotient topology}
\begin{mydefinition}
Let $X$ and $Y$ be topological spaces; let $p:X\to Y$ be a surjective map. $p$ is said to be a \textbf{\emph{quotient map}} if
$$U\underset{\text{open}}{\subset}Y \iff p^{-1}(U)\underset{\text{open}}{\subset}X.$$
\end{mydefinition}

\begin{mydefinition}
Let $X$ and $Y$ be topological spaces; let $p:X\to Y$ be surjective. A subset $C\subset X$ is \textbf{\emph{saturated}} if
$$f^{-1}f(C)=C$$
\end{mydefinition}

\begin{myproposition}
TFAE:
\begin{enumerate}[label={(\alph*)}]
\item $p:X\to Y$ is a quotient map.
\item $p$ is continuous and maps saturated open sets of $X$ to open sets of $Y$.
\item $p$ is continuous and maps saturated closed sets of $X$ to closed sets of $Y$.
\end{enumerate}    
\end{myproposition}

\begin{mydefinition}
A map $f$ from a space to a space is \textbf{\emph{open(closed) map}} if $f$ maps each open(closed) sets in domain to open(closed) sets in codomain.
\end{mydefinition}

\begin{myproposition}
If $p:X\to Y$ is a surjective continuous map that is either open or closed, then $p$ is a quotient map.
\end{myproposition}

\begin{mydefinition}
Let $X$ be a space, $A$ a set, and $p:X\to A$ surjective map. There exists exactly one topology $\mathcal{T}$ on $A$ relative to which $p$ is a quotient map. It is called the \textbf{\emph{quotient topology}} induced by $p$. $\mathcal{T}=\{U\in P(A):p^{-1}(A)\underset{\text{open}}{\subset}X\}$.
\end{mydefinition}

\begin{mydefinition}
Let
\begin{enumerate}[label={(\alph*)}]
\item $X$ be a topological space;
\item $X^*$ be a partition of $X$ into disjoint subsets whose uinon is $X$;
\item $p:X\to X^*$ be the surjective map that carries each point of $X$ to the element of $X^*$ containing it.
\end{enumerate}
In the quotient topology induced by $p$, the space $X^*$ is called a \textbf{\emph{quotient space}} of $X$.
\end{mydefinition}
This can be viewed from a different perspective. For $U\subset X^*$, $p^{-1}(U)$ is a collection of equivalence classes whose union is open in $X$.

\begin{mytheorem}
Let $p:X\to Y$ be a quotient map; let $A$ be a subspace of $X$ that is saturated with respect to $p$; let $q:A\to p(A)$ be the map obtained by restricting $p$.
\begin{enumerate}[label={(\alph*)}]
\item If $A$ is either open or closed in $X$, then $q$ is quotient map.
\item If $p$ is either an open map or a closed map, then $q$ is a quotient map. 
\end{enumerate}
\end{mytheorem}
\begin{proof}
Munkres p140
\end{proof}
