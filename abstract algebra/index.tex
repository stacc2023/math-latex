\documentclass{article}
\usepackage[landscape, margin=0.5cm]{geometry}
\usepackage{multicol}

\usepackage{helvet} % Use Helvetica font (similar to Arial)
\renewcommand{\familydefault}{\sfdefault} % 기본 글꼴을 sans-serif로 설정

\usepackage{amsmath, amsthm}

% Define a new theorem style
\newtheoremstyle{definition}
{1em} % Space above
{0.5em} % Space below
{\normalfont} % Body font
{} % Indent amount
{\bfseries} % Theorem head font
{} % Punctuation after theorem head
{0cm} % Space after theorem head
{\llap{\thmname{#1} \thmnumber{#2}\hskip1em}\thmnote{#3\vspace{0.5em}\newline}}

% Apply the new style to the definition environment
% \theoremstyle{definition}
\theoremstyle{definition}


% Define a custom definition environment
\newtheorem{definition}{Definition}[section]
\newtheorem{theorem}[definition]{Theorem}
\newtheorem{example}[definition]{Example}
\newtheorem{exercise}{Exercise}[section]
\newtheorem*{remark}{Remark}
\newtheorem*{corollary}{Corollary}

% Customize the section style
\usepackage{titlesec}
% \titleformat{\section}[block]
% {\normalfont\large\bfseries}
% {\llap{Section \thesection.0\hskip1em}}{0pt}{}

\renewcommand{\thesubsection}{\arabic{subsection}}
\titleformat{\subsection}[block]
{\normalfont\normalsize\bfseries}
{\llap{Concept \thesubsection\hskip1em}}{0pt}{\large}



% enumerate
\usepackage{enumitem}
% \alph*: 소문자 알파벳 (a, b, c, ...)
% \Alph*: 대문자 알파벳 (A, B, C, ...)
% \roman*: 소문자 로마 숫자 (i, ii, iii, ...)
% \Roman*: 대문자 로마 숫자 (I, II, III, ...)

% math font
\usepackage{amsfonts}


% Redefine the proof environment to match the theorem style
\makeatletter
\renewenvironment{proof}[1][(pf)]{%
    \par
    \pushQED{\qed}%
    \normalfont\topsep3pt\relax
    % \normalfont \topsep6\p@\@plus6\p@\relax
    \trivlist
    \item[\llap{\bfseries#1\hskip0em}]
    \leftskip=2.5em
    \def\forward{\item[\llap{\bfseries($\Rightarrow$)\hskip0em}]}
    \def\backward{\item[\llap{\bfseries($\Leftarrow$)\hskip0em}]}
    \newcommand{\step}[1]{\@ifnextchar\bgroup{\step@binary{##1}}{\step@single{##1}}}
    \newcommand{\step@binary}[2]{\item[\llap{(##1) $\Rightarrow$ (##2)\hskip0em}]}
    \newcommand{\step@single}[1]{\item[\llap{(##1)\hskip0em}]}
}{%
    \popQED\endtrivlist\@endpefalse
}
\makeatother

\usepackage{xcolor} % color
\definecolor{softred}{RGB}{239, 41, 41} % softred


\makeatletter
\newenvironment{hardproof}[1][\textcolor{softred}{(pf)}]{%
    \par
    \pushQED{\qed}%
    \normalfont\topsep3pt\relax
    % \normalfont \topsep6\p@\@plus6\p@\relax
    \trivlist
    \item[\llap{\bfseries#1\hskip0em}]
    \leftskip=2.5em
    \def\forward{\item[\llap{\bfseries($\Rightarrow$)\hskip0em}]}
    \def\backward{\item[\llap{\bfseries($\Leftarrow$)\hskip0em}]}
    \newcommand{\step}[1]{\@ifnextchar\bgroup{\step@binary{##1}}{\step@single{##1}}}
    \newcommand{\step@binary}[2]{\item[\llap{(##1) $\Rightarrow$ (##2)\hskip0em}]}
    \newcommand{\step@single}[1]{\item[\llap{(##1)\hskip0em}]}
}{%
    \popQED\endtrivlist\@endpefalse
}
\makeatother


% hyperlink
\usepackage{hyperref}

\newcommand{\diam}{\text{diam }}


% because
\usepackage{amssymb}


% box
\usepackage[most]{tcolorbox}

\newtcolorbox{notebox}[1][]{
    colback=white!0,
    colframe=black,
    sharp corners,
    boxrule=1pt,
    valign=top,
    left=5pt,
    #1
}

%integral
\usepackage{amsmath}

\def\upint{\mathchoice%
    {\mkern13mu\overline{\vphantom{\intop}\mkern7mu}\mkern-20mu}%
    {\mkern7mu\overline{\vphantom{\intop}\mkern7mu}\mkern-14mu}%
    {\mkern7mu\overline{\vphantom{\intop}\mkern7mu}\mkern-14mu}%
    {\mkern7mu\overline{\vphantom{\intop}\mkern7mu}\mkern-14mu}%
  \int}
\def\lowint{\mkern3mu\underline{\vphantom{\intop}\mkern7mu}\mkern-10mu\int}

%Riemann integral
\usepackage{mathrsfs}

% footnote horizontal
\usepackage[para]{footmisc}

\usepackage[most]{tcolorbox}
\usepackage{xcolor}

\definecolor{mintgreen}{HTML}{E0F8E0}
\definecolor{skyblue}{HTML}{D6EBFF}
\definecolor{peach}{HTML}{FFE5B4}


\newtcolorbox[use counter=definition, number within=section]{mydefinition}[1][]{
  colback=peach,        % 배경색
  colframe=peach,       % 경계선 색상 (배경색과 동일하게 설정하여 경계선을 없앰)
  fonttitle=\bfseries,         % 제목의 글씨체를 볼드체로
  title={\thetcbcounter \, Definition%
         \ifstrempty{#1}{}{:\ #1}}, % 제목 서식 설정, 인수 #1이 비어 있지 않으면 ": #1" 추가
  boxrule=0pt,                 % 경계선 두께 설정 (0pt로 설정하여 경계선 없음)
  sharp corners,               % 모서리를 뾰족하게
  coltitle=black,              % 제목 색상
  colbacktitle=peach,      % 제목 배경색 (본문 배경색과 동일)
  enhanced,                    % 박스의 기타 그래픽적 요소들 향상
  toptitle=5pt,
  left=5pt,
  right=5pt,
  bottom=5pt,
}


\newtcolorbox[use counter=definition, number within=section]{mytheorem}[1][]{
  colback=skyblue,
  colframe=skyblue,
  fonttitle=\bfseries,
  title={\thetcbcounter \, Theorem%
  \ifstrempty{#1}{}{:\ #1}},
  boxrule=0pt,
  sharp corners,
  coltitle=black,
  colbacktitle=skyblue,
  enhanced,                    % 박스의 기타 그래픽적 요소들 향상
  toptitle=5pt,
  left=5pt,
  right=5pt,
  bottom=5pt,
}

\newtcolorbox[use counter=definition, number within=section]{mylemma}[1][]{
    colback=mintgreen,
    colframe=mintgreen,
    fonttitle=\bfseries,
    title={\thetcbcounter \, Lemma%
    \ifstrempty{#1}{}{:\ #1}},
    boxrule=0pt,
    sharp corners,
    coltitle=black,
    colbacktitle=mintgreen,
    enhanced,                    % 박스의 기타 그래픽적 요소들 향상
    toptitle=5pt,
    left=5pt,
    right=5pt,
    bottom=5pt,
}

\newtcolorbox[use counter=definition, number within=section]{myproposition}[1][]{
    colback=mintgreen,
    colframe=mintgreen,
    fonttitle=\bfseries,
    title={\thetcbcounter \, Proposition%
    \ifstrempty{#1}{}{:\ #1}},
    boxrule=0pt,
    sharp corners,
    coltitle=black,
    colbacktitle=mintgreen,
    enhanced,                    % 박스의 기타 그래픽적 요소들 향상
    toptitle=5pt,
    left=5pt,
    right=5pt,
    bottom=5pt,
}

\newtcolorbox[use counter=definition, number within=section]{mycorollary}[1][]{
    colback=mintgreen,
    colframe=mintgreen,
    fonttitle=\bfseries,
    title={\thetcbcounter \, Corollary%
    \ifstrempty{#1}{}{:\ #1}},
    boxrule=0pt,
    sharp corners,
    coltitle=black,
    colbacktitle=mintgreen,
    enhanced,                    % 박스의 기타 그래픽적 요소들 향상
    toptitle=5pt,
    left=5pt,
    right=5pt,
    bottom=5pt,
}

\newtheoremstyle{definition}
{0.5em} % Space above
{0.5em} % Space below
{\normalfont} % Body font
{} % Indent amount
{\bfseries} % Theorem head font
{} % Punctuation after theorem head
{1em} % Space after theorem head
{}

\titleformat{\section}[block]
{\normalfont\large\bfseries}
{Section \thesection. }{0pt}{}

\renewcommand{\thesubsection}{\arabic{subsection}}
\titleformat{\subsection}[block]
{\normalfont\normalsize\bfseries}
{}{0pt}{\large}

\makeatletter
\renewenvironment{proof}[1][\textbf{\proofname)}]{%
    \par
    \pushQED{\qed}%
    \normalfont\topsep3pt\relax
    % \normalfont \topsep6\p@\@plus6\p@\relax
    \trivlist
    \item #1
    \def\forward{\item[\llap{\bfseries($\Rightarrow$)\hskip0em}]}
    \def\backward{\item[\llap{\bfseries($\Leftarrow$)\hskip0em}]}
    \newcommand{\step}[1]{\@ifnextchar\bgroup{\step@binary{##1}}{\step@single{##1}}}
    \newcommand{\step@binary}[2]{\item[\llap{(##1) $\Rightarrow$ (##2)\hskip0em}]}
    \newcommand{\step@single}[1]{\item[\llap{(##1)\hskip0em}]}
}{%
    \popQED\endtrivlist\@endpefalse
}
\makeatother

% img
\usepackage{graphicx}
\usepackage{wrapfig}
\usepackage{capt-of}

\begin{document}
\begin{multicols}{2}
\tableofcontents

\section{Ring}
\subsection{Lecture 0904}

\begin{mydefinition}
[Ring]
A \textbf{\emph{ring}} $R$ is an abelian group $\langle R, + \rangle $ which has another operation $\cdot$ such that
\begin{enumerate}[label={(\alph*)}]
\item $'\cdot'$ is associative.
\item $(a+b)\cdot c=a\cdot c + b\cdot c$ and $c\cdot(a+b)=c\cdot a+c\cdot b$ for all $a,b,c\in R$.
\end{enumerate}
\end{mydefinition}

\begin{example}
$G= \langle \mathbb{Z},+ R=\rangle\to \langle \mathbb{Z},+,\cdot  \rangle$: a ring
\end{example}

\begin{mydefinition}
$R$ is called a \textbf{\emph{commutative ring}} if for every $a,b\in R$, $a\cdot b=b\cdot a$.
\end{mydefinition}

\begin{example}
$R=\langle (M^\infty (\mathbb{Z})),+,\cdot  \rangle $ is NOT commutative.
\end{example}

\begin{mydefinition}
An element $1_R\in R$ is called a \textbf{\emph{unity}} if for every $a\in R$ $a\cdot a=a\cdot 1=a$.
\end{mydefinition}

\begin{myproposition}
$1_R$ is unique in $R$.
\end{myproposition}

\begin{mydefinition}
An element $u\in R$ is called a \textbf{\emph{unit element}} if there exists $u'\in R$ such that $u\cdot u'=1_R$.
\end{mydefinition}

\begin{mydefinition}
Suppose $R,R'$ are two rings. $f:R\to R'$ is called a \textbf{\emph{ring homomorphism}} if
\begin{enumerate}[label={(\alph*)}]
\item $f(a+_Rb)=f(a)+_{R'}f(b)$.
\item $f(a\cdot_R b)=f(a)\cdot_{R'} f(b)$.
\end{enumerate}
\end{mydefinition}

\begin{remark}
The ring homomorphism $f$ is injective if $\ker f=\{r\in R|f(r)=0_{R'}\}=\{0_R\}$.
\end{remark}

\begin{mydefinition}
Any subgroup $I$ of $R$ is called \textbf{\emph{ideal}} of $R$ if
\begin{enumerate}[label={(\alph*)}]
\item $I\subset R$
\item $R\cdot I=I\cdot R\subset I$
\end{enumerate}
\end{mydefinition}

\begin{example}
Suppose $f$ is a ring homomorphism. Then $\forall \alpha\in R$, $\forall r\in \ker f$, $\alpha \cdot r\in R$ and $f(\alpha \cdot r)=f(\alpha)f(r)=0$. That is, $\ker f$ is an ideal of $R$.
\end{example}

\begin{mydefinition}
An nonzero element $\alpha\in R$ is called a \textbf{\emph{zero divisor}} if there exists a nonzero element $\beta\in R$ such that $\alpha \cdot \beta = 0$.
\end{mydefinition}

\begin{mydefinition}
$R$ is called an \textbf{\emph{integral domain}} if $R$ is a commutative ring with $1_R$ and no zero devisors.
\end{mydefinition}

\begin{myproposition}
[Cancellation laws]
Suppose $R$ is a integral domain. for all nonzero element $a,b\in R$, $ac=ab \implies a(b-c)=0 \implies b=c$.
\end{myproposition}

\subsection{Lecture 0909}
\begin{mydefinition}
By a \textbf{\emph{division ring}}, we mean a ring $R$ with unity $1$ such that $\forall r\in R$, $r$ has multicative inverse $r'$ such that $r\cdot r'=1$.
\end{mydefinition}
\begin{myproposition}
If a ring $R$ is a division ring, then $r$ has no zero divisor.
\end{myproposition}

\begin{mydefinition}
By a \textbf{\emph{field}}, we mean a ring $R$ such that $R$ is a integral domain and that every nonzero element $r\in R$ has an inverse in $R$.
\end{mydefinition}

\begin{myproposition}
A commutative division ring is a field.
\end{myproposition}

\begin{mytheorem}
\begin{enumerate}[label={(\alph*)}]
\item Every finite integral domain is a feild.
\item Every finite division ring is a field.
\end{enumerate}
\end{mytheorem}
\begin{proof}
(a) Let $R$ be a finite integral domain. We may assume $R=\{a_1,a_2,\dots,a_n\}$. Given a nonzero element $r\in R$, define $\psi: R\to R$ by $\psi(a_i)=ra_i$. For $\alpha,\beta \in R$, $r\alpha=r\beta\implies r(\alpha-\beta)=0$, which means $a=b$ since $R$ has no zero divisors. Then there exists a $1\leq i\leq n$ such that $ra_i=1$.
\end{proof}

\begin{mydefinition}
$n\in \mathbb{Z}^+$ is called the \textbf{\emph{characteristic}} of $R$ if there exists a number s.t. $n\cdot r=0$ for all $r\in R$. $n$ is the smallest such number.
\end{mydefinition}

\begin{mycorollary}
If $R$ is a ring with unity $1$, the characteristic of $R$ is the smallest number $n\cdot 1=0$.
\end{mycorollary}

\begin{mylemma}
$\mathbb{Z}^n$ is an integral domain $\iff$ $n$ is a prime number.
\end{mylemma}
\begin{proof}
$(\Rightarrow)$ If $n$ is not prime, then $n=n_1n_2$ for some $1<n_1,n_2<n$. It implies that $n_1n_2 \equiv 0 \mod n$, i.e., $n_1,n_2$ are zero divisors.\\
$(\Leftarrow)$ Suppose, for contradiction, $\mathbb{Z}_p$ is not integral domain. There exists some nonzero elements $n_1,n_2\in \mathbb{Z}_p$ such that $n_1n_2\equiv 0 \mod p$. Then $p|n_1$ or $p|n_2$, which means $n_1 = 0$ or $n_2 = 0$.
\end{proof}

\begin{mylemma}
Let $\mathcal{R}\subset \mathbb{Z}$ be an nonempty ideal of $\mathbb{Z}$. Then $\mathcal{R}=n \mathbb{Z} = \{na:a\in \mathbb{Z}\}$ for some $n\in \mathbb{Z}^+$.
\end{mylemma}
\begin{proof}
Let $r$ be the smallest number in $\mathcal{R}$. By the division algorithm, $r=nq+s$ for some $n,q,s$. since $r,nq\in \mathcal{R}$, $s=0$. Thus $r=nq\in n \mathbb{Z} \implies \mathcal{R}\subset n \mathbb{Z}$. The other direction is trivial.
\end{proof}

\begin{mytheorem}
$\mathbb{Z}_n$ is a field $\iff$ $n$ is a prime number $p$. In this case,
\begin{enumerate}[label={(\alph*)}]
\item $\mathbb{Z}_p$ is called a \textbf{\emph{finite field}}.
\item Every finite field $F$ contains $\mathbb{Z}_p$ for some prime $p$.
\end{enumerate}
\end{mytheorem}
\begin{proof}
Consider a ring homomorphism: $\psi: \mathbb{Z}\to F$ given by
$$\psi(n)=\begin{cases}
n\cdot 1_F=1_F+\dots+1_F & \text{for }n>0\\
-|n|\cdot 1_F=(-1_f)+\cdots + (-1_F) & \text{for }n<0\\
0_F & \text{for }n=0
\end{cases}$$
Then $n \mathbb{Z}=\ker \psi\subset \mathbb{Z}=R \implies \mathbb{Z}/\ker \psi \simeq \psi(\mathbb{Z})\subset F$ for some $n\in \mathbb{Z}$. Since the field $F$ has no zero devisor, $n$ must be a prime number.
\end{proof}

\begin{mylemma}
For all nonzero number $a\in \mathbb{Z}$, $a^{p-1}\equiv 1 \mod p$.
\end{mylemma}
\begin{proof}

\end{proof}

\begin{notebox}
$\mathbb{Z}_n^*=\{\overline{r}\in \mathbb{Z}_n:(r,n)=1\}$. $|\mathbb{Z}_n^*|=\psi(n)=\text{ the number of }\{r\in \mathbb{N}:(r,n)=1\}$.
\end{notebox}

\begin{mytheorem}
$\mathbb{Z}_n^*$ forms a group with $\cdot$.
\end{mytheorem}
\begin{proof}
since $(a,n)=1$, for some $\alpha,\beta\in \mathbb{Z}$, $\alpha a + \beta n = 1 \implies \alpha a \equiv 1 \mod n \implies \overline{a}$ has inverse $\overline{\alpha}$ in $\mathbb{Z}_n^*$.
\end{proof}

\begin{mytheorem}
If $a\in \mathbb{Z}$, $(a,n)=1$, then $a^{\psi(1)}\equiv 1\mod n$.
\end{mytheorem}
\begin{proof}
$G=\mathbb{Z}_n^*$ is a group by the pervious thm, of order $|G|=\psi(n)$, by the Lagrange thm, $a^{|G|}=a^{\psi(n)}\equiv 1 \mod n$.
\end{proof}

\begin{mytheorem}
If $(a,m)=1$, then $ax\equiv b\mod n$ has a unique solution in $\mathbb{Z}_m$.
\end{mytheorem}
\begin{proof}
By the previous thm, $\mathbb{Z}_m^*$ is a group. $\overline{a}\in \mathbb{Z}_m^* \implies \exists \overline{a'}\in \mathbb{Z}_m^*$ s.t. $\overline{a}\cdot \overline{a'}=\overline{1}\mod m$. Then, $x\equiv \overline{a'}\cdot \overline{b}$ is the unique solution.
\end{proof}

\begin{mytheorem}
Let $d=\gcd(a,m)$. Then $ax\equiv b\mod m$ has a solutuin $\iff d|b$. In this case, there are $d$-solutions.
\end{mytheorem}
\begin{proof}
$(\Rightarrow)$ Assume $a=da_1$, $m=dm_1$. 
$ax\equiv b\mod m$ has a solution $x_0$ $\implies$ $ax_0\equiv b\mod m$ $\implies$ $m|(ax_0-b) \implies d|dm_1|da_1x_0-b\implies d|b$. \\
$(\Leftarrow)$ $d|b\implies b=b_1d$, $a=a_1d$, $m=m_1d$. $ax\equiv b\mod m \implies a_1dx\equiv bd \mod md \implies a_1x\equiv b_1 \mod m_1 \implies$ there exists a unique solution in $\mathbb{Z}_{m_1}$. Consider a ring-homo $\phi: \mathbb{Z}_m\to \mathbb{Z}_{m_1}$. Then $\phi^{-1}(x) = \{x,x+m_1,\dots,x+(d-1)m_1\}$.
\end{proof}

\begin{notebox}
HW1: p189 \#11,13,15,17,19,21,22,27~30
\end{notebox}

\subsection{Lecture 0911}
\begin{mytheorem}
[Division algorithm]
Let $F[x]=\{\sum_{i=0}^{n}a_ix^i\mid a_i\in F,n\geq 0\}$.
\begin{enumerate}[label={(\alph*)}]
\item For all $f(x),g(x)\in F[x]$, $f(x)=q(x)g(x)+r(x)$ for $\deg(r(x))<\deg(g(x))$.
\item For all $f(x)\in F[x]$, '$a$' is a root of $f(x)$, i.e., $f(a)=0\iff (x-a)\mid f(x)\iff f(x)=(x-a)f'(x)$.
\item Every finite multiplicative group of a field $F$ must be cyclic.
\item (Eisenstein crieteria) For all $f(x)=a_nx^n+a_{n-1}x^{n-1}+\dots +a_1x+a_0\in Q[x]$ is \textbf{\emph{iwed??}} if there exists a prime $p\in \mathbb{Z}$ such that $p\nmid a_n$, $p\mid a_i$ for $0\leq i\leq n-1$, and $p^2\nmid a_0$. 
\end{enumerate}
Recall that $f(x)$ is \textbf{\emph{rdasd??}} in $F[x]$ if $f(x)=f_1(x)f_2(x)$ in $F[x]$ and $\deg f_i(x)$ is non-zero or $f_i(x)$ is not constant. Otherwise, $f(x)$ is said to be \textbf{\emph{????}}.
\end{mytheorem}

\begin{notebox}
HW2: p207 \# 12,14,15,16,17,24,27, p218 \# 14,16,34,35,36,37
\end{notebox}


\clearpage
\end{multicols}
\end{document}
