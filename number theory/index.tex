\documentclass{article}
\usepackage[landscape, margin=0.5cm]{geometry}
\usepackage{multicol}

\usepackage{helvet} % Use Helvetica font (similar to Arial)
\renewcommand{\familydefault}{\sfdefault} % 기본 글꼴을 sans-serif로 설정

\usepackage{amsmath, amsthm}

% Define a new theorem style
\newtheoremstyle{definition}
{1em} % Space above
{0.5em} % Space below
{\normalfont} % Body font
{} % Indent amount
{\bfseries} % Theorem head font
{} % Punctuation after theorem head
{0cm} % Space after theorem head
{\llap{\thmname{#1} \thmnumber{#2}\hskip1em}\thmnote{#3\vspace{0.5em}\newline}}

% Apply the new style to the definition environment
% \theoremstyle{definition}
\theoremstyle{definition}


% Define a custom definition environment
\newtheorem{definition}{Definition}[section]
\newtheorem{theorem}[definition]{Theorem}
\newtheorem{example}[definition]{Example}
\newtheorem{exercise}{Exercise}[section]
\newtheorem*{remark}{Remark}
\newtheorem*{corollary}{Corollary}

% Customize the section style
\usepackage{titlesec}
% \titleformat{\section}[block]
% {\normalfont\large\bfseries}
% {\llap{Section \thesection.0\hskip1em}}{0pt}{}

\renewcommand{\thesubsection}{\arabic{subsection}}
\titleformat{\subsection}[block]
{\normalfont\normalsize\bfseries}
{\llap{Concept \thesubsection\hskip1em}}{0pt}{\large}



% enumerate
\usepackage{enumitem}
% \alph*: 소문자 알파벳 (a, b, c, ...)
% \Alph*: 대문자 알파벳 (A, B, C, ...)
% \roman*: 소문자 로마 숫자 (i, ii, iii, ...)
% \Roman*: 대문자 로마 숫자 (I, II, III, ...)

% math font
\usepackage{amsfonts}


% Redefine the proof environment to match the theorem style
\makeatletter
\renewenvironment{proof}[1][(pf)]{%
    \par
    \pushQED{\qed}%
    \normalfont\topsep3pt\relax
    % \normalfont \topsep6\p@\@plus6\p@\relax
    \trivlist
    \item[\llap{\bfseries#1\hskip0em}]
    \leftskip=2.5em
    \def\forward{\item[\llap{\bfseries($\Rightarrow$)\hskip0em}]}
    \def\backward{\item[\llap{\bfseries($\Leftarrow$)\hskip0em}]}
    \newcommand{\step}[1]{\@ifnextchar\bgroup{\step@binary{##1}}{\step@single{##1}}}
    \newcommand{\step@binary}[2]{\item[\llap{(##1) $\Rightarrow$ (##2)\hskip0em}]}
    \newcommand{\step@single}[1]{\item[\llap{(##1)\hskip0em}]}
}{%
    \popQED\endtrivlist\@endpefalse
}
\makeatother

\usepackage{xcolor} % color
\definecolor{softred}{RGB}{239, 41, 41} % softred


\makeatletter
\newenvironment{hardproof}[1][\textcolor{softred}{(pf)}]{%
    \par
    \pushQED{\qed}%
    \normalfont\topsep3pt\relax
    % \normalfont \topsep6\p@\@plus6\p@\relax
    \trivlist
    \item[\llap{\bfseries#1\hskip0em}]
    \leftskip=2.5em
    \def\forward{\item[\llap{\bfseries($\Rightarrow$)\hskip0em}]}
    \def\backward{\item[\llap{\bfseries($\Leftarrow$)\hskip0em}]}
    \newcommand{\step}[1]{\@ifnextchar\bgroup{\step@binary{##1}}{\step@single{##1}}}
    \newcommand{\step@binary}[2]{\item[\llap{(##1) $\Rightarrow$ (##2)\hskip0em}]}
    \newcommand{\step@single}[1]{\item[\llap{(##1)\hskip0em}]}
}{%
    \popQED\endtrivlist\@endpefalse
}
\makeatother


% hyperlink
\usepackage{hyperref}

\newcommand{\diam}{\text{diam }}


% because
\usepackage{amssymb}


% box
\usepackage[most]{tcolorbox}

\newtcolorbox{notebox}[1][]{
    colback=white!0,
    colframe=black,
    sharp corners,
    boxrule=1pt,
    valign=top,
    left=5pt,
    #1
}

%integral
\usepackage{amsmath}

\def\upint{\mathchoice%
    {\mkern13mu\overline{\vphantom{\intop}\mkern7mu}\mkern-20mu}%
    {\mkern7mu\overline{\vphantom{\intop}\mkern7mu}\mkern-14mu}%
    {\mkern7mu\overline{\vphantom{\intop}\mkern7mu}\mkern-14mu}%
    {\mkern7mu\overline{\vphantom{\intop}\mkern7mu}\mkern-14mu}%
  \int}
\def\lowint{\mkern3mu\underline{\vphantom{\intop}\mkern7mu}\mkern-10mu\int}

%Riemann integral
\usepackage{mathrsfs}

% footnote horizontal
\usepackage[para]{footmisc}

\usepackage[most]{tcolorbox}
\usepackage{xcolor}

\definecolor{mintgreen}{HTML}{E0F8E0}
\definecolor{skyblue}{HTML}{D6EBFF}
\definecolor{peach}{HTML}{FFE5B4}


\newtcolorbox[use counter=definition, number within=section]{mydefinition}[1][]{
  colback=peach,        % 배경색
  colframe=peach,       % 경계선 색상 (배경색과 동일하게 설정하여 경계선을 없앰)
  fonttitle=\bfseries,         % 제목의 글씨체를 볼드체로
  title={\thetcbcounter \, Definition%
         \ifstrempty{#1}{}{:\ #1}}, % 제목 서식 설정, 인수 #1이 비어 있지 않으면 ": #1" 추가
  boxrule=0pt,                 % 경계선 두께 설정 (0pt로 설정하여 경계선 없음)
  sharp corners,               % 모서리를 뾰족하게
  coltitle=black,              % 제목 색상
  colbacktitle=peach,      % 제목 배경색 (본문 배경색과 동일)
  enhanced,                    % 박스의 기타 그래픽적 요소들 향상
  toptitle=5pt,
  left=5pt,
  right=5pt,
  bottom=5pt,
}


\newtcolorbox[use counter=definition, number within=section]{mytheorem}[1][]{
  colback=skyblue,
  colframe=skyblue,
  fonttitle=\bfseries,
  title={\thetcbcounter \, Theorem%
  \ifstrempty{#1}{}{:\ #1}},
  boxrule=0pt,
  sharp corners,
  coltitle=black,
  colbacktitle=skyblue,
  enhanced,                    % 박스의 기타 그래픽적 요소들 향상
  toptitle=5pt,
  left=5pt,
  right=5pt,
  bottom=5pt,
}

\newtcolorbox[use counter=definition, number within=section]{mylemma}[1][]{
    colback=mintgreen,
    colframe=mintgreen,
    fonttitle=\bfseries,
    title={\thetcbcounter \, Lemma%
    \ifstrempty{#1}{}{:\ #1}},
    boxrule=0pt,
    sharp corners,
    coltitle=black,
    colbacktitle=mintgreen,
    enhanced,                    % 박스의 기타 그래픽적 요소들 향상
    toptitle=5pt,
    left=5pt,
    right=5pt,
    bottom=5pt,
}

\newtcolorbox[use counter=definition, number within=section]{myproposition}[1][]{
    colback=mintgreen,
    colframe=mintgreen,
    fonttitle=\bfseries,
    title={\thetcbcounter \, Proposition%
    \ifstrempty{#1}{}{:\ #1}},
    boxrule=0pt,
    sharp corners,
    coltitle=black,
    colbacktitle=mintgreen,
    enhanced,                    % 박스의 기타 그래픽적 요소들 향상
    toptitle=5pt,
    left=5pt,
    right=5pt,
    bottom=5pt,
}

\newtcolorbox[use counter=definition, number within=section]{mycorollary}[1][]{
    colback=mintgreen,
    colframe=mintgreen,
    fonttitle=\bfseries,
    title={\thetcbcounter \, Corollary%
    \ifstrempty{#1}{}{:\ #1}},
    boxrule=0pt,
    sharp corners,
    coltitle=black,
    colbacktitle=mintgreen,
    enhanced,                    % 박스의 기타 그래픽적 요소들 향상
    toptitle=5pt,
    left=5pt,
    right=5pt,
    bottom=5pt,
}

\newtheoremstyle{definition}
{0.5em} % Space above
{0.5em} % Space below
{\normalfont} % Body font
{} % Indent amount
{\bfseries} % Theorem head font
{} % Punctuation after theorem head
{1em} % Space after theorem head
{}

\titleformat{\section}[block]
{\normalfont\large\bfseries}
{Section \thesection. }{0pt}{}

\renewcommand{\thesubsection}{\arabic{subsection}}
\titleformat{\subsection}[block]
{\normalfont\normalsize\bfseries}
{}{0pt}{\large}

\makeatletter
\renewenvironment{proof}[1][\textbf{\proofname)}]{%
    \par
    \pushQED{\qed}%
    \normalfont\topsep3pt\relax
    % \normalfont \topsep6\p@\@plus6\p@\relax
    \trivlist
    \item #1
    \def\forward{\item[\llap{\bfseries($\Rightarrow$)\hskip0em}]}
    \def\backward{\item[\llap{\bfseries($\Leftarrow$)\hskip0em}]}
    \newcommand{\step}[1]{\@ifnextchar\bgroup{\step@binary{##1}}{\step@single{##1}}}
    \newcommand{\step@binary}[2]{\item[\llap{(##1) $\Rightarrow$ (##2)\hskip0em}]}
    \newcommand{\step@single}[1]{\item[\llap{(##1)\hskip0em}]}
}{%
    \popQED\endtrivlist\@endpefalse
}
\makeatother

% img
\usepackage{graphicx}
\usepackage{wrapfig}
\usepackage{capt-of}

\begin{document}
\begin{multicols}{2}
\tableofcontents

\section{Mathematical induction}
\begin{mytheorem}
[well-ordering principle(axiom)]
Every nonempty set $S$ of non-negative integers contains a least element, i.e., there exists $a\in S$ such that $a\leq x$ for all $x\in S$.

Consider $S'=\{x-b:x\in S\}$. Then $S'$ must have a least element, say $y$. Then $y+b$ is the least element of $S$.
\end{mytheorem}

\begin{mytheorem}
[Archimedean property]
If $a,b$ be positive integers. Then there exists a positive integer $n$ such that $an\geq b$.
\end{mytheorem}
\begin{proof}
Suppose by contradiction, $\forall k\in \mathbb{N}, ak<b$. Consider that $S=\{b-ak|k\in \mathbb{N}\}$ consists of integers large than or equal to $1$. By well-ordering principle, $S$ contains the minimal element, say $b-am$ ($m\in \mathbb{N}$). Then $0<b-a(m+1)<b-am \implies b-a(m+1)\in S$, leading to a contradiction.
\end{proof}

\begin{mytheorem}
[First principle of finite induction]
Suppose that $S$ is a set of integers satisfying
\begin{enumerate}[label={(\alph*)}]
\item $1\in S$;
\item if $k\in S$, then $k+1\in S$.
\end{enumerate}
Then $S$ is the set of all positive integers.
\end{mytheorem}
\begin{proof}
Let $T=\mathbb{N}\backslash S$. Suppose that $T$ is not empty. By the well-ordering principle, $T$ contains a least element, say $n$. Then $n\geq 2$ ($\because 1\in S \implies 1\notin T$), and $n-1\notin T \implies n-1\in S$. By (b), $n\in S$, leading to a contradiction.
\end{proof}


\begin{example}
Show that for all $n\in \mathbb{N}$,
$$1+3+5+\dots +(2n-1)=n^2$$
\end{example}
\begin{proof}
$n=1 \implies 1=1^2=1$.
Suppose the assertion holds for $n=k$.
Then $1+3+\dots+(2k-1)+(2k+1)=k^2+2k+1=(k+1)^2$, holds for $n=k+1$.
\end{proof}

\begin{remark}
[Second principle of finite induction]
(b) can be repalced by the condition (b') If $k$ is a positive integer and $1,2,\dots,k \in S$, then $k+1\in S$.
\end{remark}

\begin{mytheorem}
[Second principle of finite induction]
Let $S$ be a sef of positive integer satisfying (a),(b'). Then $S=\mathbb{N}$
\end{mytheorem}
The proof is similar

\begin{example}
Let $\{a_n\}$ be a sequence with $a_1=1$, $a_2=2$, $a_3=3$ and $a_n=a_{n-1}+a_{n-2}+a_{n-3}$ for all $n\geq 4$. Show that $a_n<2^n$ for all $n\in \mathbb{N}$
\end{example}
The proof is an exercise.

\begin{mytheorem}
[The binomial theorem]
\begingroup
\everymath{\displaystyle}
$\begin{pmatrix}
n\\k
\end{pmatrix}= _nC_k$
\endgroup
: the number of ways of choosing $k$ numbers in $\{1,2,\dots\}$.
\begingroup
\everymath{\displaystyle}
$$\begin{pmatrix}
n\\
k
\end{pmatrix}=\frac{n!}{k!(n-k)!}\quad (0\leq k\leq n)$$
\endgroup
\end{mytheorem}

\begin{mytheorem}
[Pascal's rule]
\begingroup
\everymath{\displaystyle}
$$
\begin{pmatrix}
n\\
k-1
\end{pmatrix}+
\begin{pmatrix}
n\\
k
\end{pmatrix}=
\begin{pmatrix}
n+1\\
k
\end{pmatrix}
\quad (1\leq k\leq n)
$$
\endgroup
\end{mytheorem}

\begin{proof}
LHS $\displaystyle = \frac{n!}{(k-1)!(n-k+1)!} + \frac{n!}{k!(n-k)!}=\frac{n!}{(k-1)!(n-k)!}\left(\frac{1}{n-k+1}+\frac{1}{k}\right)=\frac{n!}{(k-1)!(n-k)!}(\frac{n+1}{(n-k+1)k})=\frac{(n+1)!}{k!(n-k+1)!}$.
\end{proof}

\begin{mytheorem}
[Binomial expansion]
Complete exansion of $(a+b)^n\quad (n\geq 1)$ into a sum of poners of $a$ and $b$.
\begin{align}
    (a+b)^1 &= a+b\\
    (a+b)^2 &= a^2+2ab+b^2\\
    (a+b)^3 &= a^3+3a^2b+3ab^2+b^3\\
    (a+b)^4 &= a^4+4a^3b+6a^2b^2+4ab^3+b^4
\end{align}
\end{mytheorem}

\begin{mytheorem}
For $n\geq 1$, positive integer
\begingroup
\everymath{\displaystyle}
$$(a+b)^n=\sum_{k=0}^{n}\begin{pmatrix}
n\\
k
\end{pmatrix}
a^kb^{n-k}
$$
\endgroup
\end{mytheorem}
\begin{proof}
Use induction on $n$. $n=1$: clear. Suppose the equality holds for $n=m$. Then
\begin{align}
    (a+b)(a+b)^m&=\left(\sum_{k=0}^{m}(m,k)a^kb^{m-k} \right)(a+b)\\  
    &= \sum_{k=0}^{m}(m,k)a^{k+1}b^{m-k}+\sum_{k=0}^{m}(m,k)a^kb^{m-k+1}\\
    &=(m,k)a^{m+1}+\sum_{k=0}^{m-1}(m,k)a^{k+1}b^{m-k}+(m,0)b^{m+1}+\sum_{k=1}^{m}(m,k)a^kb^{m-k+1}\\
    &=a^{m+1}+\sum_{k=1}^{m}\left((m,k-1)+(m,k)\right)a^kb^{m-k+1}+b^{m+1}\\
    &=a^{m+1}+\sum_{k=1}^{m}(m+1,k)a^kb^{m-k+1}+b^{m+1}\\
    &=\sum_{k=0}^{m+1}(m+1,k)a^kb^{m+1-k}
\end{align}
\end{proof}

\begin{example}
For $n\geq 1$,
\begin{enumerate}[label={(\alph*)}]
\item $\displaystyle \sum_{k=0}^{n}(n,k)=2^n$
\item $\displaystyle \sum_{k=0}^{n}(-1)^k(n,k)=0$
\item $\displaystyle (n,1)+(n,3)+\dots =(n,0)+(n,2)+(n,4)+\dots = 2^{n-1}$
\end{enumerate}
\end{example}
The proof is trivial.

\subsection{Lecture 0909}
\section{Divisibility theory in the integers}
\subsection{The division algorithm}
\begin{mytheorem}
Suppose $a,b\in \mathbb{Z}$ and $b>0$. Then there exists unique integers $q$ and $r$ such that $a=qb+r$ and $0\leq r<b$. $q$,$r$ is called the \textbf{\emph{quotient}} and \textbf{\emph{remainder}} respectively.
\end{mytheorem}
\begin{proof}
Let $S=\{a-kb:k\in \mathbb{Z},a-kb\geq 0\}$. Clearly $S\neq \varnothing$. By the well-ordering pinrciple, $S$ has the minimal element, say $r$. Assume $r=a-qb$. We claim that $0\leq r<b$. Suppose to the contrary $r\geq b$. Then $0\leq a-(q+1)b=a-qb-b< a-qb$, whci contradicts to the minimality of $r$. The existance of such $q,r$ follows. To prove uniqueness, suppose $a=q_1b+r_1=q_2b+r_2$. Then $|b(q_1-q_2)|=|r_1-r_2|$. Since $|r_1-r_2|<b$ and $b|(r_1-r_2)$, $r_1=r_2$.
\end{proof} 

\begin{mycorollary}
Suppose $a,b$ are integers and $b\neq 0$. Then there exists unique $q,r\in \mathbb{Z}$ such that
\begin{enumerate}[label={(\alph*)}]
\item $a=qb+r$.
\item $0\leq r<|b|$.
\end{enumerate} 
\end{mycorollary}
\begin{proof}
The case $b>0$ holds by the previous theorem. If $b<0$, there exists $q',r'\in \mathbb{Z}$ such that $a=q'|b|+r'=b(-q')+r'=bq+r$ where $0\leq r'<|b|$. By the uniqueness of $q,r$, $q' = -q$ and $r=r'$.
\end{proof}

\begin{example}
Let $a\in \mathbb{Z}$ and $b=2$. Division algorithm says that $a$ is of the form $2q$ or $2q+1$.
\end{example}

\begin{example}
$a^2$ leaves the remainder 0 or 1 when divided by 4(remainder 0,1,or 4 when divided by 8). $a=2q\implies a^2=4q^2$. $a=2q+1 \implies a^2=4q^2+4q+1=4q(q+1)+1=8k+1$.
\end{example}

\begin{example}
$a^4$ is of the form $5k$ or $5k+1$.
$a = 5q+r \quad 0\leq r \leq 4$. $a^4=(5q=r)^4=(5q)^4+\dots+{4\choose4} r^4$. $r^4 \equiv 1 \mod 5$.
\end{example}

\begin{example}
More generally, if $p$ is a prime, then $a^{r-1}$ is of the form $pk$ or $pk+1$ (Fermat's little theorem).
\end{example}

\section{The greatest common divisor}
\begin{mydefinition}
An integer $b$ is said to be \textbf{\emph{divisible}} by $a\neq 0$ if $\exists c\in \mathbb{Z}$ such that $b=ac$, we write $a\mid b$. We write $x\nmid y$ to mean $b$ is not divisible by $a$.
\end{mydefinition}

\begin{mytheorem}
Suppose $a,b,c\in \mathbb{Z}$ and $a\neq 0$. Then
\begin{enumerate}[label={(\alph*)}]
\item $a\mid 0$, $1\mid b$, and $a\mid a$.
\item $a\mid 1 \iff a=1$ or $a=-1$.
\item $a\mid b$, $c\mid d \implies ac\mid db$.
\item $a\mid b$, $b\mid c \implies a\mid c$.
\item $a\mid b$, $b\mid a \iff a=b$ or $a=-b$.
\item $a\mid b$, $b\neq 0 \implies |a|\leq |b|$.
\item $a\mid x_1, a\mid x_2,\dots, a\mid x_n$, then $a\mid (b_1x_1+b_2x_2+\dots+b_nx_n)$ with $b_i\in \mathbb{Z}$
\end{enumerate}
\end{mytheorem}
\begin{proof}
(f) $b=ac\quad (c\neq 0)$. Then $|b|=|ac|=|a||c|\geq |a|$. 
\end{proof}

\begin{mydefinition}
Suppose $a,b$ are integers.
An integer $d$ such that $d\mid a$ and $d\mid b$ is called a \textbf{\emph{common divisor}} of $a$ and $b$.
\end{mydefinition}

\begin{mydefinition}
Suppose $a,b\in \mathbb{Z}$ and $a\neq 0$ or $b\neq 0$.
Then the greatest common divisor(g.c.d) $a$ and $b$, denoted by $gcd(a,b)$, is the positive integer $d$ satisfying
\begin{enumerate}[label={(\alph*)}]
\item $d\mid a$ and $d\mid b$.
\item If $c\mid a$ and $c\mid b$, then $c\leq d$.
\end{enumerate}
\end{mydefinition}

\begin{remark}
For any nonzero interger $b$, there are only finitely many divisors. Therefore, $gcd(a,b)$ exists if $a\neq 0$ or $b\neq 0$.
\end{remark}

\begin{mytheorem}
Suppose $a,b\in \mathbb{Z}$, $a\neq 0$ or $b\neq 0$, and $d=gcd(a,b)$. Then there exists $x,y\in \mathbb{Z}$ such that $ax+by=d$.
\end{mytheorem}
\begin{proof}
Consider a set $S=\{am+bn\mid m,n\in \mathbb{Z}, am+bn>0\}$. By the Archimedean property, $S\neq \varnothing$. By the well-ordering principle, the smallest element $\exists s\in S$. Claim $s=d=gcd(a,b)$. To prove $s$ is a common divisor, use division algorithm: $a=qs+r$ with $0\leq r<s$. If $r\neq 0$, then $r=a-qs=a-q(ax+by)=a(1-qx)+b(-qy)\in S$. It is a contradiction, so $r=0$ and $s\mid a$. Similarly, $s\mid b$. Let $c$ be common divisor of $a$ and $b$. Then $c\mid ax+by=s \implies |c|\leq |s|=s$. Thus $s=d=\gcd(a,b)$. 
\end{proof}

\begin{mycorollary}
Supppose $a,b\in \mathbb{Z}$, $a\neq 0$ or $b\neq 0$. Then $T=\{ax+by\mid x,y\in \mathbb{Z}\}$ is precisely the set of all multiples of $\gcd(a,b)=d$.
\end{mycorollary}
\begin{proof}
$T'=\{dn\mid n\in \mathbb{Z}\}$. WTS: $T=T'$\\
$(\supset)$ $d=am+bk$ for $m,k\in \mathbb{Z}$. $dn = a(mn)+b(kn)\in T$.\\
$(\subset)$ $\forall ax+by\in T$ is a multiple of $d$ $\implies$ $T\subset T'$.
\end{proof}

\begin{mydefinition}
Suppose $a,b\in \mathbb{Z}$ and $a\neq 0$ or $b\neq 0$. Then $a,b$ are said to be \textbf{\emph{relatively prime}} if $\gcd(a,b)=1$
\end{mydefinition}

\begin{mytheorem}
suppose $a,b\in \mathbb{Z}$ and $a\neq 0$ or $b\neq 0$.
$$\gcd(a,b)=1\iff \exists m,n\in \mathbb{Z} \text{ such that }1=am+ba$$
\end{mytheorem}
\begin{myproposition}
$(\Rightarrow)$ Follows from the previous thm.\\
$(\Leftarrow)$ If $c$ is a common divisor of $a$ and $b$, then $c\mid 1$ and $c=+-1\leq 1 \implies \gcd(a,b)=1$. 
\end{myproposition}

\begin{mycorollary}
If $\gcd(a,b)=d$, then $\gcd(\frac{a}{d},\frac{b}{d})=1$.
\end{mycorollary}
\begin{proof}
$\exists m,n\in \mathbb{Z}$, $am+bn=d \implies (\frac{a}{d}m+(\frac{b}{d})n=1) \implies \gcd(\frac{a}{d},\frac{b}{d})=1$.
\end{proof}

\begin{mycorollary}
If $a\mid c$, $b\mid c$ and $\gcd(a,b)=1$, then $ab\mid c$.
\end{mycorollary}
\begin{proof}
$\gcd(a,b)=1 \implies \exists m,n\in \mathbb{Z}$ such that $1=am+bn$. Then $c=1c=(am+bn)c=acm+cbn=abrm+absm$ ($\because c=br=as$ for some $r,s\in \mathbb{Z})$.
\end{proof}

\begin{mytheorem}
[Euclid's lemma]
If $a\mid bc$ and $gcd(a,b)=1$, then $a\mid c$.
\end{mytheorem}
\begin{proof}
$1=am+bn$ for some $m,n\in \mathbb{Z}$. Then $c=1c= c(am+bn)=acm+bcn$, which is divisible by $a$.
\end{proof}

\begin{mytheorem}
[Alternative definition of gcd]
Suppose $a,b\in \mathbb{Z}$ and $a\neq 0$ or $b\neq 0$. For a positive integer $d$, $d=gcd(a,b) \iff$
\begin{enumerate}[label={(\alph*)}]
\item $d\mid a$ and $d\mid d$
\item If $c\mid a$ and $c\mid b$, then $c\mid d$.
\end{enumerate}
\end{mytheorem}
\begin{proof}
$(\Leftarrow)$ Clear as (b) implies that if $c\mid a$ and $c\mid b$, then $c\leq |c|\leq |d|=d$.\\
$(\Rightarrow)$ If $c\mid a$ and $c\mid b$, then $c\mid ax+by$ for any $x,y\in \mathbb{Z}$. $\exists m,n\in \mathbb{Z}$ such that $am+bn=d$. 
\end{proof}

\begin{remark}
$\gcd(a,b)=\gcd(c,d) \iff$
\begin{enumerate}[label={(\alph*)}]
\item Every common divisor of $a$ and $b$ is a common divisor of $c$ and $d$.
\item Every common divisor of $c$ and $d$ is a comoon divisor of $a$ and $b$.
\end{enumerate} 
\end{remark}

\begin{example}
$\gcd(a,b)=\gcd(a,-b)=\gcd(-a,b)=\gcd(-a,-b)$.
\end{example}

\section{Euclidean algorithm}

\begin{example}
\begin{enumerate}[label={(\alph*)}]
\item[]
\item Find $\gcd(a,b)$
\item $\gcd(a,b)=d$. We know $\exists x,y\in \mathbb{Z}$ such that $d=ax+by$.
\end{enumerate}
The Euclidean algorithm gives us $x,y$.
Suppose $a,b\in \mathbb{Z}$ and $a\neq 0$ or $b\neq 0$. $\gcd(a,b)=gcd(|a|,|b|)$. We may assume $a,b>0$. $a=q_1b+r_1$ for ($0\leq r_1<b$). If $r_1=0$, $\gcd(a,b)=b$. If $r_1\neq 0$, then for $b$ and $r_1$, $b=q_2r_1+r_2\quad(0\leq r_2<r_1)$. The algorithm terminate when we arrive at $r_{n+1}=0$. Then $gcd(a,b)=r_n$.        
\end{example}

\begin{mytheorem}
If $a=qb+r$, then $\gcd(a,b)=\gcd(b,r)$.
\end{mytheorem}
\begin{proof}
$c\mid a,b \implies c\mid r \implies c\mid b,r$. Thus, $\gcd(a,b)\mid \gcd(b,r)$. Similarly if $c'\mid b,r$, then $c'\mid a=qb+r$.
\end{proof}

\begin{example}
    \begin{align}
        \gcd(a,b)=r_n=& r_{n-2}-q_nr_{n-1}\\
        =&r_{n-2}-q_n(r_{n-3}-q_{n-1}r_{n-2})\\
        &\vdots\\
        =&a(X) + b(Y)
    \end{align}
\end{example}

\subsection{Lecture 0911}


\clearpage
\end{multicols}
\end{document}
