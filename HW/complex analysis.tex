\documentclass{article}
\usepackage[landscape, margin=0.5cm]{geometry}
\usepackage{multicol}

\usepackage{helvet} % Use Helvetica font (similar to Arial)
\renewcommand{\familydefault}{\sfdefault} % 기본 글꼴을 sans-serif로 설정

\usepackage{amsmath, amsthm}

% Define a new theorem style
\newtheoremstyle{definition}
{1em} % Space above
{0.5em} % Space below
{\normalfont} % Body font
{} % Indent amount
{\bfseries} % Theorem head font
{} % Punctuation after theorem head
{0cm} % Space after theorem head
{\llap{\thmname{#1} \thmnumber{#2}\hskip1em}\thmnote{#3\vspace{0.5em}\newline}}

% Apply the new style to the definition environment
% \theoremstyle{definition}
\theoremstyle{definition}


% Define a custom definition environment
\newtheorem{definition}{Definition}[section]
\newtheorem{theorem}[definition]{Theorem}
\newtheorem{example}[definition]{Example}
\newtheorem{exercise}{Exercise}[section]
\newtheorem*{remark}{Remark}
\newtheorem*{corollary}{Corollary}

% Customize the section style
\usepackage{titlesec}
% \titleformat{\section}[block]
% {\normalfont\large\bfseries}
% {\llap{Section \thesection.0\hskip1em}}{0pt}{}

\renewcommand{\thesubsection}{\arabic{subsection}}
\titleformat{\subsection}[block]
{\normalfont\normalsize\bfseries}
{\llap{Concept \thesubsection\hskip1em}}{0pt}{\large}



% enumerate
\usepackage{enumitem}
% \alph*: 소문자 알파벳 (a, b, c, ...)
% \Alph*: 대문자 알파벳 (A, B, C, ...)
% \roman*: 소문자 로마 숫자 (i, ii, iii, ...)
% \Roman*: 대문자 로마 숫자 (I, II, III, ...)

% math font
\usepackage{amsfonts}


% Redefine the proof environment to match the theorem style
\makeatletter
\renewenvironment{proof}[1][(pf)]{%
    \par
    \pushQED{\qed}%
    \normalfont\topsep3pt\relax
    % \normalfont \topsep6\p@\@plus6\p@\relax
    \trivlist
    \item[\llap{\bfseries#1\hskip0em}]
    \leftskip=2.5em
    \def\forward{\item[\llap{\bfseries($\Rightarrow$)\hskip0em}]}
    \def\backward{\item[\llap{\bfseries($\Leftarrow$)\hskip0em}]}
    \newcommand{\step}[1]{\@ifnextchar\bgroup{\step@binary{##1}}{\step@single{##1}}}
    \newcommand{\step@binary}[2]{\item[\llap{(##1) $\Rightarrow$ (##2)\hskip0em}]}
    \newcommand{\step@single}[1]{\item[\llap{(##1)\hskip0em}]}
}{%
    \popQED\endtrivlist\@endpefalse
}
\makeatother

\usepackage{xcolor} % color
\definecolor{softred}{RGB}{239, 41, 41} % softred


\makeatletter
\newenvironment{hardproof}[1][\textcolor{softred}{(pf)}]{%
    \par
    \pushQED{\qed}%
    \normalfont\topsep3pt\relax
    % \normalfont \topsep6\p@\@plus6\p@\relax
    \trivlist
    \item[\llap{\bfseries#1\hskip0em}]
    \leftskip=2.5em
    \def\forward{\item[\llap{\bfseries($\Rightarrow$)\hskip0em}]}
    \def\backward{\item[\llap{\bfseries($\Leftarrow$)\hskip0em}]}
    \newcommand{\step}[1]{\@ifnextchar\bgroup{\step@binary{##1}}{\step@single{##1}}}
    \newcommand{\step@binary}[2]{\item[\llap{(##1) $\Rightarrow$ (##2)\hskip0em}]}
    \newcommand{\step@single}[1]{\item[\llap{(##1)\hskip0em}]}
}{%
    \popQED\endtrivlist\@endpefalse
}
\makeatother


% hyperlink
\usepackage{hyperref}

\newcommand{\diam}{\text{diam }}


% because
\usepackage{amssymb}


% box
\usepackage[most]{tcolorbox}

\newtcolorbox{notebox}[1][]{
    colback=white!0,
    colframe=black,
    sharp corners,
    boxrule=1pt,
    valign=top,
    left=5pt,
    #1
}

%integral
\usepackage{amsmath}

\def\upint{\mathchoice%
    {\mkern13mu\overline{\vphantom{\intop}\mkern7mu}\mkern-20mu}%
    {\mkern7mu\overline{\vphantom{\intop}\mkern7mu}\mkern-14mu}%
    {\mkern7mu\overline{\vphantom{\intop}\mkern7mu}\mkern-14mu}%
    {\mkern7mu\overline{\vphantom{\intop}\mkern7mu}\mkern-14mu}%
  \int}
\def\lowint{\mkern3mu\underline{\vphantom{\intop}\mkern7mu}\mkern-10mu\int}

%Riemann integral
\usepackage{mathrsfs}

% footnote horizontal
\usepackage[para]{footmisc}

\usepackage[most]{tcolorbox}
\usepackage{xcolor}

\definecolor{mintgreen}{HTML}{E0F8E0}
\definecolor{skyblue}{HTML}{D6EBFF}
\definecolor{peach}{HTML}{FFE5B4}


\newtcolorbox[use counter=definition, number within=section]{mydefinition}[1][]{
  colback=peach,        % 배경색
  colframe=peach,       % 경계선 색상 (배경색과 동일하게 설정하여 경계선을 없앰)
  fonttitle=\bfseries,         % 제목의 글씨체를 볼드체로
  title={\thetcbcounter \, Definition%
         \ifstrempty{#1}{}{:\ #1}}, % 제목 서식 설정, 인수 #1이 비어 있지 않으면 ": #1" 추가
  boxrule=0pt,                 % 경계선 두께 설정 (0pt로 설정하여 경계선 없음)
  sharp corners,               % 모서리를 뾰족하게
  coltitle=black,              % 제목 색상
  colbacktitle=peach,      % 제목 배경색 (본문 배경색과 동일)
  enhanced,                    % 박스의 기타 그래픽적 요소들 향상
  toptitle=5pt,
  left=5pt,
  right=5pt,
  bottom=5pt,
}


\newtcolorbox[use counter=definition, number within=section]{mytheorem}[1][]{
  colback=skyblue,
  colframe=skyblue,
  fonttitle=\bfseries,
  title={\thetcbcounter \, Theorem%
  \ifstrempty{#1}{}{:\ #1}},
  boxrule=0pt,
  sharp corners,
  coltitle=black,
  colbacktitle=skyblue,
  enhanced,                    % 박스의 기타 그래픽적 요소들 향상
  toptitle=5pt,
  left=5pt,
  right=5pt,
  bottom=5pt,
}

\newtcolorbox[use counter=definition, number within=section]{mylemma}[1][]{
    colback=mintgreen,
    colframe=mintgreen,
    fonttitle=\bfseries,
    title={\thetcbcounter \, Lemma%
    \ifstrempty{#1}{}{:\ #1}},
    boxrule=0pt,
    sharp corners,
    coltitle=black,
    colbacktitle=mintgreen,
    enhanced,                    % 박스의 기타 그래픽적 요소들 향상
    toptitle=5pt,
    left=5pt,
    right=5pt,
    bottom=5pt,
}

\newtcolorbox[use counter=definition, number within=section]{myproposition}[1][]{
    colback=mintgreen,
    colframe=mintgreen,
    fonttitle=\bfseries,
    title={\thetcbcounter \, Proposition%
    \ifstrempty{#1}{}{:\ #1}},
    boxrule=0pt,
    sharp corners,
    coltitle=black,
    colbacktitle=mintgreen,
    enhanced,                    % 박스의 기타 그래픽적 요소들 향상
    toptitle=5pt,
    left=5pt,
    right=5pt,
    bottom=5pt,
}

\newtcolorbox[use counter=definition, number within=section]{mycorollary}[1][]{
    colback=mintgreen,
    colframe=mintgreen,
    fonttitle=\bfseries,
    title={\thetcbcounter \, Corollary%
    \ifstrempty{#1}{}{:\ #1}},
    boxrule=0pt,
    sharp corners,
    coltitle=black,
    colbacktitle=mintgreen,
    enhanced,                    % 박스의 기타 그래픽적 요소들 향상
    toptitle=5pt,
    left=5pt,
    right=5pt,
    bottom=5pt,
}

\newtheoremstyle{definition}
{0.5em} % Space above
{0.5em} % Space below
{\normalfont} % Body font
{} % Indent amount
{\bfseries} % Theorem head font
{} % Punctuation after theorem head
{1em} % Space after theorem head
{}

\titleformat{\section}[block]
{\normalfont\large\bfseries}
{Section \thesection. }{0pt}{}

\renewcommand{\thesubsection}{\arabic{subsection}}
\titleformat{\subsection}[block]
{\normalfont\normalsize\bfseries}
{}{0pt}{\large}

\makeatletter
\renewenvironment{proof}[1][\textbf{\proofname)}]{%
    \par
    \pushQED{\qed}%
    \normalfont\topsep3pt\relax
    % \normalfont \topsep6\p@\@plus6\p@\relax
    \trivlist
    \item #1
    \def\forward{\item[\llap{\bfseries($\Rightarrow$)\hskip0em}]}
    \def\backward{\item[\llap{\bfseries($\Leftarrow$)\hskip0em}]}
    \newcommand{\step}[1]{\@ifnextchar\bgroup{\step@binary{##1}}{\step@single{##1}}}
    \newcommand{\step@binary}[2]{\item[\llap{(##1) $\Rightarrow$ (##2)\hskip0em}]}
    \newcommand{\step@single}[1]{\item[\llap{(##1)\hskip0em}]}
}{%
    \popQED\endtrivlist\@endpefalse
}
\makeatother

% img
\usepackage{graphicx}
\usepackage{wrapfig}
\usepackage{capt-of}

\begin{document}
\begin{multicols}{2}

\section{HW 1}

\begin{exersise}
[Lemma 1.2]
Let $z,w\in \mathbb{C}$.
\begin{enumerate}[label={(\alph*)}]
\item $\overline{z+w}=\overline{z}+\overline{w}$ and $\overline{z\cdot w}=\overline{z}\cdot \overline{w}$.
\item $z+\overline{z}=2\text{Re}(z)$ and $z-\overline{z}=i2\text{Im}(z)$.
\item $|\overline{z}|=|z|$ and $|z\cdot w|=|z||w|$.
\item $|\text{Re}(z)|\leq |z|$ and $|\text{Im}(z)|\leq |z|$.
\end{enumerate}
\end{exersise}
\begin{proof}
Let $z=z_1+z_2i$ and $w=w_1+w_2i$.
\begin{enumerate}[label={(\alph*)}]
\item $\overline{z+w}=\overline{(z_1+z_2i)+(w_1+w_2i)}=\overline{(z_1+w_1)+(z_2+w_2)i}=(z_1+w_1)-(z_2+w_2)i=(z_1-z_2i)+(w_1-w_2i)=\overline{z}+\overline{w}$.\\
$\overline{z\cdot w}=\overline{(z_1+z_2i)\cdot(w_1+w_2i)}=\overline{(z_1w_1-z_2w_2)+(z_1w_2+z_2w_1)i}=(z_1w_1-z_2w_2)-(z_1w_2+z_2w_1)i=(z_1-z_2i)\cdot (w_1-w_2i)=\overline{z}\cdot \overline{w}$.
\item $z+\overline{z}=(z_1+z_2i)+(z_1-z_2i)=2z_1=2\text{Re}(z)$\\
$z-\overline{z}=(z_1+z_2i)-(z_1-z_2i)=2z_2i=2i\text{Im}(z)$.
\item $|\overline{z}|=\sqrt{\overline{z}\cdot \overline{\overline{z}}}=\sqrt{\overline{z}\cdot z}=\sqrt{z\cdot \overline{z}}=|z|$.\\
$|z\cdot w|^2 = z\cdot w\cdot \overline{z}\cdot \overline{w} = z\cdot \overline{z}\cdot w\cdot \overline{w} = |z|^2|w|^2 \implies |z\cdot w|=|z||w|$.
\item $|\text{Re}(z)|^2=z_1^2 \leq z_1^2+z_2^2 = |z|^2 \implies |\text{Re}(z)|\leq |z|$.\\
$|\text{Im}(z)|^2=z_2^2 \leq z_1^2+z_2^2 = |z|^2 \implies |\text{Im}(z)|\leq |z|$.
\end{enumerate}
\end{proof}

\begin{exersise}
[Lemma 1.5]
If $f,g\in C(U)$, then $f+g,fg\in C(U)$.
\end{exersise}
\begin{proof}
\begin{enumerate}[label={(\alph*)}]
\item[$(f+g)$] Given $\epsilon>0$ and fixed $x_0\in U$, choose open subsets $V_1,V_2\subset U$ such that 
$$x_0\in V_1 \implies \sup_{x\in V_1}|f(x)-f(x_0)|<\frac{\epsilon}{2}$$ and $$x_0\in V_2 \implies \sup_{x\in V_2}|f(x)-f(x_0)|<\frac{\epsilon}{2}.$$
Then 
\begin{align*}
    x_0\in V=V_1\cap V_2 &\Rightarrow \sup_{x_0\in V}|(f+g)(x)-(f+g)(x_0)|\\
    &\leq \sup_{x_0\in V}|f(x)-f(x_0)|+\sup_{x_0\in V}|g(x)-g(x_0)|\\
    & < \frac{\epsilon}{2} + \frac{\epsilon}{2} = \epsilon
\end{align*}
\item[$(fg)$] Given $\epsilon>0$ and fixed $x_0\in U$, choose open subsets $V_1,V_2\subset U$ such that
$$x_0\in V_1 \implies \sup_{x_0\in V_1}|f(x)-f(x_0)|<\frac{\epsilon}{4|g(x_0)|}$$
and
$$x_0\in V_2 \implies \sup_{x_0\in V_2}|g(x)-g(x_0)|<\frac{\epsilon}{2|f(x_0)|}$$
and 
$$x_0\in V_2 \implies \sup_{x_0\in V_2}|g(x)-g(x_0)|< |g(x_0)|\text{, i.e., }\sup_{x_0\in V_2}|g(x)|<2|g(x_0)|.$$
Then
\begin{align*}
    x_0\in V=V_1\cap V_2 &\Rightarrow \sup_{x_0\in V}|fg(x)-fg(x_0)|\\
    &=\sup_{x_0\in V}|f(x)g(x)-f(x_0)g(x)+f(x_0)g(x)-f(x_0)g(x_0)|\\
    &\leq\sup_{x_0\in V}|g(x)(f(x)-f(x_0))|+\sup_{x_0\in V}|f(x_0)(g(x)-g(x_0))|\\
    &< 2|g(x_0)|\cdot \frac{\epsilon}{4|g(x_0)|} + |f(x_0)|\cdot + \frac{\epsilon}{2|f(x_0)|}\\
    &=\frac{\epsilon}{2}+\frac{\epsilon}{2}=\epsilon
\end{align*}
\end{enumerate}
\end{proof}

\begin{exersise}
[Lemma 1.7]
\begin{enumerate}[label={(\alph*)}]
\item[]
\item Let $f(z)=1/z$. Then $f\in H(\mathbb{C}-\{0\})$ and $f'(z)=-1/z^2$.
\item Let $f(z)=\overline{z}$. Then $f$ is nowhere differentiable.
\end{enumerate}
\end{exersise}
\begin{proof}
\begin{enumerate}[label={(\alph*)}]
\item Let $z_0\in U\backslash \{0\}$ be given. Since the set $U=\mathbb{C}\backslash \{0\}$ is open, there is a neighborhood $V_0$ of $z_0$ contained in $U\backslash \{0\}$. Let $z_1\in V_0$ be a point. Since $V_0$ is open, there is a neighborhood of $z_1$ does not contain $0$. Therefore, we can apply definition 1.7 for every $z_1\in V_0$:
$$\lim_{z\to z_1}\frac{f(z)-f(z_1)}{z-z_1}=\lim_{z\to z_1}\frac{\frac{1}{z}-\frac{1}{z_1}}{z-z_1}=\lim_{z\to z_1}\frac{-1}{zz_1}$$
Now we claim that
$$\lim_{z\to z_1}\frac{-1}{zz_1}=\frac{-1}{z_1^2}.$$
Let $\epsilon>0$ be given, and assume $\delta \leq \frac{1}{2}|z_1|$. Then we have
$$\frac{1}{2}|z_1|<|z|<\frac{3}{2}|z_1|.$$
We observe that
$$\left|\frac{-1}{zz_1}-\frac{-1}{z_1^2}\right|=\left|\frac{-z_1+z}{zz_1^2}\right|<\frac{\delta}{\frac{1}{2}|z_1|z_1^2}$$
But we want
$$\frac{\delta}{\frac{1}{2}|z_1|z_1^2} \leq \epsilon.$$
Consequently, it is sufficient to set $\displaystyle \delta=\min\left\{\frac{1}{2}|z_1|, \frac{1}{2}\epsilon|z_1|^3\right\}$.
\item Let $z_0\in \mathbb{C}$ be given, and suppose $z_0=x_0+y_0i$ and $z=x+yi$ for $z\neq z_0$. Then 
$$\lim_{z\to z_0} \frac{\overline{z}-\overline{z_0}}{z-z_0}
=\lim_{z\to z_0} \frac{(x-x_0)+(-y+y_0)i}{(x-x_0)+(y-y_0)i}
$$
If we fix $x=x_0$, then the value of limit is $-1$, or if we fix $y=y_0$, then the value of limit is $1$, which implies that the limit does not converge.
\end{enumerate}
\end{proof}

\begin{exersise}
[Corollary 1.9]
If $f$ is differentiable at $c$, then $f$ is continuous at $c$.
\end{exersise}
\begin{proof}
By the theorem 1.8, $f$ can be expressed as a sum or product of continuous functions at $c$.
By the lemma 1.5, it must also be continuous at $c$.
\end{proof}

\begin{exersise}
[Lemma 1.10]
If $f,g\in H(U)$, then $f+g,fg\in H(U)$ and $(f+g)'=f'+g'$ and $(fg)'=f'g+fg'$. 
\end{exersise}
\begin{proof}
Let $z_0\in U$ be given. There exists $r_f,r_g>0$ such that $f,g$ are differentiable on $D(z_0,r_f),D(z_0,r_g)$ respectively. Set $r=\min(r_f,r_s)$. We claim that $f+g$ and $fg$ is differentiable on $D(z_0,r)$. To prove this, let $z_1\in D(z_0,r)$ be given. 
For some $a_f,a_g\in \mathbb{C}$ and some $h_f,h_g:U\to \mathbb{C}$ which are continuous at $z_1$ and which are zero at $z_1$,
$$f(z) = f(z_1)+a_f(z-z_1)+h_f(z)(z-z_1)$$
$$g(z) = g(z_1)+a_g(z-z_1)+h_g(z)(z-z_1)$$
$a_f,a_g$ denote the derivative of $f$ and $g$ at $x_0$, respectively.
Then
$$(f+g)(z)= (f+g)(z_1)+(a_f+a_g)(z-z_1)+(h_f+h_z)(z)(z-z_1).$$
If we set $a_{f+g}=a_f+a_g$ and $h_{f+g}=h_f+h_z$, then it satisfies the conditions for differentiability($\because h_{f+g}(z) = 0$ at $z_1$ and continuous at $z_1$ by the lemma 1.5). On the other hand,
\begin{align*}
    (fg)(z) =& (f(z_1)+a_f(z-z_1)+h_f(z)(z-z_1))(g(z_1)+a_g(z-z_1)+h_g(z)(z-z_1))\\
    =& fg(z_1)+f(z_1)a_g(z-z_1)+f(z_1)h_g(z)(z-z_1)\\
    &+a_fg(z_1)(z-z_1)+a_fa_g(z-z_1)^2+a_fh_g(z)(z-z_1)^2\\
    &+h_f(z)g(z_1)(z-z_1)+h_f(z)a_g(z-z_1)^2+h_fh_g(z)(z-z_1)^2\\
    =& fg(z_1)+(f(z_1)a_g+a_fg(z_1))(z-z_1)\\
    &+ (f(z_1)h_g(z)+a_fa_g(z-z_1)+a_fh_g(z)(z-z_1)\\
    &\quad +h_f(z)g_(z_1)+h_f(z)a_g(z-z_1)+h_fh_g(z)(z-z_1))(z-z_1)
\end{align*}
If we set $a_{fg}=f(z_1)a_g+a_fg(z_1)$ and $h_{fg}$ equal to the third term on the right-hand side of the above equation, then, since $h_f, h_g, h_gh_f, (z-z_1)$ are all continuous at $z_1$ and have a value of zero at $z_1$, by the lemma 1.5, $h_{fg}$ satisfies the condition for differentiability.
\end{proof}

\begin{exersise}
[Lemma 1.11]
If $f:U\to \mathbb{C}$ is differentiable at $c$ and $g:f(U)\to \mathbb{C}$ is differentiable at $f(c)$, then $g\circ f$ is differentiable at $c$ and $(g\circ f)'(c)=g'(f(c))f'(c)$.
\end{exersise}
\begin{proof}
To prove this, we should first prove a lemma regarding the continuity of composition of continuous functions.
\begin{notebox}
(Lemma 1.12) Let $A,B,C$ be sets, $f:A\to B$ continuous, and $g:f(A)\to C$ continuous. Then $g\circ f$ is also continuous.
\end{notebox}
By the definition of the continuity of functions, for every open set $U\in C$, $g^{-1}(U)$ is open in $B$, and $f^{-1}(g^{-1}(U))$ is open in $A$. Therefore, the lemma holds. Let
$$f(x) = f(c) + a_f(x-c) + h_f(x)(x-c)$$
$$g(y) = g(f(c)) + a_g(y-f(c)) + h_g(y)(y-f(c))$$
Here, $a_f$ and $a_g$ denote the derivatives of $f$ and $g$ at $c$ and $f(c)$, respectively, and $h_f$ and $h_g$ denote continuous functions which are zero at $c$ and $f(c)$, respectively.
Then,
\begin{align*}
    (g\circ f)(x) =& g(f(c)) + a_g(f(x)-f(c)) + h_g(f(x))(f(x)-f(c))\\
    =&g(f(c))+a_g(f(c)+a_f(x-c)+h_f(x)(x-c)-f(c))\\
    &+h_g(f(x))(f(c)+a_f(x-c)+h_f(x)(x-c)-f(c))\\
    =&g(f(c))+a_ga_f(x-c)+(a_gh_f(x)+h_g(f(x))a_f+h_g(f(x))h_f(x))(x-c)
\end{align*}
If we set $a_{g\circ f}=a_ga_f$ and $h_{g\circ f}=a_gh_f(x)+h_g(f(x))a_f+h_g(f(x))h_f(x)$, then $h_{g\circ f}$ is continuous by the lemma 1.5 and $h_{g\circ f}(c)=0$. Then, by the theorem 1.8, $g\circ f$ is differentiable at $c$.
\end{proof}

\clearpage

% \section{HW2}
% $\varnothing$


\end{multicols}
\end{document}
